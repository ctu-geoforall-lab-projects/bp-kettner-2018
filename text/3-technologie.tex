\chapter{Použité technologie}
\label{3-technologie}
%% ML: programy -> technologie (ci neco vice obecneho nez programy)
V této kapitole budou zmíněny programy, které byly pro tvorbu
bakalářské práce využity. Patří sem hlavně programovací jazyk Python,
%% ML: plugin vznika? zkuste preformulovat
geografický informační systém QGIS, pro který zásuvný modul vzniká a s
%% ML: posledni vete vubec nerozumim, orientaci? Mate na mysli, to ze
%% jste pouzival Qt Creator pro tvorbu UI? Qt je graficky framework na
%% kterem je QGIS postaven.
ním spojenou knihovnu GDAL. Qt Project byl využit k lepší orientace ve
zdrojovém kódu zásuvného modulu.

\section{Python}
\label{sec:python}
\begin{figure}[H]
	 \centering
      \includegraphics[width=5cm]{./pictures/python-logo.png}
      \caption{Logo Python (zdroj:
\href{https://www.python.org/static/community_logos/python-logo-master-v3-TM.png}{Python.org})}
      \label{fig:python}
  \end{figure}

%% ML: druhou cast souveti prepiste anebo uplne vynechte
Za autora programovacího jazyka Python je považován Guido vam Roosum,
%% ML: drou vetu prepiste do cestiny, pocatek ... byl ... na
který byl v roli hlavního programátora. Počátek vývoje jazyka Python
byl v roce 1990 na Stichtig Mathematisch Centrum v Nizozemí. Základy
%% ML: osloveni Guido zni familiarne, odkud jste text cerpal?
vychází z programovacího jazyka ABC, který Guido do té doby
vyvíjel. Python je volně dostupný včetně standardních knihoven,
dokumentace a zdrojových kódů. V roce 2001 vznikla nezisková
organizace Python Software Foundation, která je vlastníkem veškerých
intelektuálních materiálů souvisejících s programovacím jazykem
%% ML: poznamka pod carou: velmi zjednodusena definice open source
Python. Spravuje open source\footnote{Jedná se o volně dostupné
  technologie} licence Pythonu od verze 2.1 a výš. Zároveň se stará o
%% ML: posledni veta (pod), ktera se na vyvoji jazyka take podilela a pod...
ochranou známku jazyka. Jedním ze sponzorů neziskové organizace je i
společnost Digital Creations, pod kterou vývoj jazyka také probíhal.

Python účinně a efektivně pracuje s vysokoúrovňovými datovými
typy. Syntaxe jazyka a dynamické typy z něj dělají vhodný nástroj pro
%% ML: mate v seznamu zkratek?, zde nemate rozepsanu
psaní skriptů a rychlý vývoj aplikací (\zk{RAD}). Tento jazyk si
%% ML: cteni syntaxe nezni cesky,..
Snadno oblíbí začátečníci, pro které je čtení a porozumění syntaxe
%% ML: co znamena ``interpret''
relativně snadné. Další výhodou je spustitelnost interpretru na velkém
množství platforem zahrnující Linux, Windows, MacOS a
DOS. \cite{ucebnicepython, python}

%obrazek: https://www.raspberrypi.org/documentation/usage/python/images/python-logo.png
%zdroje:https://www.python.org/ a https://i.iinfo.cz/files/root/k/Ucebnice_jazyka_Python.pdf
\section{QGIS}

\begin{figure}[H]
	 \centering
      \includegraphics[height=4cm]{./pictures/qgis-logo.jpg}
      \caption{Logo QGIS (zdroj:
\href{https://euipo.europa.eu/copla/image/CJ4JX4FZVCC523YA2TMALSKFLFPOWZHPVHYMP5QREVP2BOXHB3PCM7RCOZR6TEIMWNCQDAB6N25VA}{qgis.org})}
      \label{fig:qgis}
  \end{figure}
  
Jedná se o volně dostupný geografický informační systém (\zk{GIS}),
který slouží pro práci s geodaty \footnote{Data s prostorovou a
  atributovou složkou, která se vztahují ke konkrétnímu místu na
  %% ML: druhou vetu prepiste, aby znela vice cesky
   Zemi.}. Program je open source s licencí pod GNU General Public
License. QGIS je oficiálním projektem Open Source Geospatial
Foundation(\zk{OSGeo}) a je spustitelný na nejužívanějších operačních
systémech jako Windows, Linux, MacOS. Samotný systém je napsaný v
jazyce C++ a jeho výhodou je velké množství nejrůznějších rozšíření
(zásuvných modulů), které je možné snadno doinstalovat. Zásuvné moduly
mohou být napsané nejen v C++, ale také v programovacím jazyce
Python. QGIS podporuje velké množství formátů -- rastrových,
vektorových i databázových. Aktuálně nejnovější verze je 2.18.15
nazvaná \textit{Las Palmas} a vydaná dne 8.12.2017.

Vývoj systému začal roku 2002 Garym Shermanem, ještě pod názvem
\textit {Quantum GIS} (toto označení zůstalo až do verze 2.0). V roce
%% ML: tuto informaci mate na zacatku odstavce, staci jednou
2007 se projekt stal součástí Open Source Geospatial Foundation
(\zk{OSGeo}). \cite{qgis_official, qgis_wiki_en, qgis_wiki_cz}
 
%zdroj: https://www.qgis.org/en/site/, https://cs.wikipedia.org/wiki/QGIS, https://en.wikipedia.org/wiki/QGIS
%obrazek: https://euipo.europa.eu/copla/image/CJ4JX4FZVCC523YA2TMALSKFLFPOWZHPVHYMP5QREVP2BOXHB3PCM7RCOZR6TEIMWNCQDAB6N25VA
\section{QGIS VFK Plugin}
\begin{figure}[H]
	 \centering
      \includegraphics[height=5cm]{./pictures/Qgisvfkplugin.png}
      \caption{Ukázka prostředí pluginu (zdroj:
        %% ML: freegis.fsv.cvut.cz
\href{http://freegis.fsv.cvut.cz/wiki/images/4/4b/Qgisvfkplugin-screenshot-05.png}{fsv.cvut.cz})}
      \label{fig:qgis_vfk_plugin}
  \end{figure}

Jde o zásuvný modul (anglicky plugin) pro geografický informační systém
%% ML: predlozka ``z'' neni v tomto kontextu moc vhodna
%% ML: predlozky mate v textu naduzivany a casto ve spatne kontextu (pod/z)
QGIS, který umožňuje práci s daty z českého katastru
%% ML: na tomto miste bych nepouzival ``uplne'', pro plugin je
%% podstatne, aby se v datech vyskytovali bloky PAR a BUD (skupina
%% NEMO), jinak nemusi byt '``uplna''
nemovitostí. Zásuvný modul pracuje s úplnými daty v takzvaném novém
výměnném formátu katastru, který je označovaný \zk{VFK} nebo
\zk{NVF}. Data ze souboru jsou čtena pomocí knihovny \zk{GDAL}. Plugin
%% ML: Pri cem jinem? ;-) Zkuste preformulovat
umožňuje vyhledávání a zobrazování informací při práci s daty katastru
nemovitostí. Ovládání je uživatelsky přívětivé a známé vzhledem k
podobnému rozhraní jako je u webových aplikací.

%% ML: nyni?
V nahraných datech je nyní možné vyhledávat: parcely, budovy, jednotky
a oprávněné osoby. Prohlížeč dat umožňuje zobrazit list vlastnictví a
další výpisy informací: o parcele, o budově, o jednotce a o oprávněné
osobě. Dále prohlížeč umožňuje zobrazení aktuálního stavu nemovitosti
na stránkách Nahlížení do katastru nemovitostí a export výpisů do
formátů HTML a zdrojového kódu LaTeXu (možnost vytvoření PDF a
PS\footnote{Programovací jazyk PostScript vyvinutý ke grafickému
  popisu tisknutelných dokumentů, výhodou je nezávislost na zařízení,
  ze kterého se tiskne, podobně jako formát PDF. \cite{PostScript}}).

%% ML: Obe vety zacinaji temer stejne: Zdrojove kody/texty, zkuste prepsat
Zdrojové texty zásuvného modulu je možné stáhnout z
\href{https://github.com/ctu-geoforall-lab/qgis-vfk-plugin}{Git
  repozitáře}. Zdrojové kódy jsou šířeny pod licencí
\href{https://raw.githubusercontent.com/ctu-osgeorel/qgis-vfk-plugin/master/LICENSE}{GNU
  GPL}.

První prototyp (verze 1.x) byl napsán v jazyce C++ a vyvinut studenty
oboru Geoinformatika Annou Kratochvílovou a Václavem Petrášem na FSv
ČVUT v Praze v roce 2012. Druhá verze 2.x byla vyvíjena v letech 2015
a 2016 studentem stejného oboru Štěpánem Bambulou a zásuvný modul byl
přepsán do jazyka Python. \citep{vfk_qgis_plugin}
%http://freegis.fsv.cvut.cz/gwiki/VFK_/_QGIS_plugin
%https://cs.wikipedia.org/wiki/PostScript
 
\section{GDAL}
\label{sec:gdal}
\begin{figure}[H]
	 \centering
      \includegraphics[height=5cm]{./pictures/gdal-logo.png}
      \caption{Logo GDAL (zdroj:
\href{https://upload.wikimedia.org/wikipedia/commons/thumb/d/df/GDALLogoColor.svg/572px-GDALLogoColor.svg.png}{gdal.org})}
      \label{fig:gdal}
  \end{figure}
  
Geospatial Data Abstraction Library (GDAL) je knihovna určená pro čtení
a zápis vektorových a rastrových formátů geodat. Jde o open source
vyvíjený pod licencí X/MIT a jako součást projektu Open Source
Geospatial Foundation (\zk{OSGeo}). Samotná knihovna je reprezentována
jedním abstraktním modelem pro rastrová data a jedním pro vektorová
data. Dále knihovna nabízí užitečné nástroje pro příkazovou řádku,
které slouží ke konverzi a zpracování dat. Od verze GDAL 2.0 je
součástí knihovny GDAL také knihovna OGR, která zajišťuje
funkcionalitu jednoduchých prvků vektorových dat.

Nejdříve byla knihovna vyvíjena Frankem Warmerdamem, od verze 1.3.2
došlo k převedení na GDAL/OGR Project Management Committee, který je
součástí \zk{OSGeo}. Knihovna je díky velké funkcionalitě často
využívána v komerční i nekomerční sféře a proto patří v \zk{GIS} mezi
hlavní volně dostupné softwary. \cite{gdal, gdal_wiki}
%obrazek: https://upload.wikimedia.org/wikipedia/commons/thumb/d/df/GDALLogoColor.svg/572px-GDALLogoColor.svg.png
%zdroje: http://www.gdal.org/, https://cs.wikipedia.org/wiki/GDAL

\section{Qt Project}

\begin{figure}[H]
	 \centering
      \includegraphics[width=3cm]{./pictures/qt-logo.png}
      \caption{Logo Qt (zdroj:
\href{https://upload.wikimedia.org/wikipedia/commons/thumb/0/0b/Qt_logo_2016.svg/578px-Qt_logo_2016.svg.png}{wikipedia.org})}
      \label{fig:qt}
  \end{figure}

%% ML: chybi Vam tu kontext: na tomto grafickem frameworku je postaven QGIS!
Qt je multiplatformní aplikační rámec (framework\footnote{Softwarová
  struktura, která slouží jako podpora při vývoji dalších nových
  softwarů}), který je určen pro vývoj aplikačního softwaru. Ten může
snadno fungovat na různých platformách s žádnými nebo jen minimálními
změnami v kódu. Qt je aktuálně vyvíjen společností \textit{The Qt
  Company} a \textit{Qt Project}.\cite{qt_wiki, qt}

%zdroj: https://en.wikipedia.org/wiki/Qt_(software)
%obrazek: https://upload.wikimedia.org/wikipedia/commons/thumb/0/0b/Qt_logo_2016.svg/578px-Qt_logo_2016.svg.png

