\chapter{Závěr}
\label{5-zaver}
%Shrnutí cíle práce

%% ML: vyskrtnete ``moji'', jasne veci neni treba opakovat
%% LK: to je :-)
%% ML: odstavec - jedno ohromne souveti, prepiste...
%% LK: prepisu
%% ML: jeste se na ten odstavec podivejte, ve vsech vetam mluvite o "rozsireni"
%% ML: zkuste jeste jednou odstavec projit a preformulovat, zaver by mel byt nejlepsi casti praci
%% LK: pomuze slovo vylepsit?
%% ML: treba, pro urychleni jsem prepsal, v druhe vete treba Upraveny
%% ML: vylepsovany jsem odstranil
Cílem bakalářské práce bylo vytvoření nové knihovny napsané v jazyce
Python a~vylepšení funkcionality stávajícího zásuvného modul QGIS
pro práci s katastrálními daty. Zásuvný modul umožňuje práci
pouze se zpoplatněnými českými katastrálními daty ve formátu
\zk{VFK}. Nové rozšíření mělo navíc umožňovat načítání neúplných
veřejně dostupných dat výměnného formátu katastru nemovitostí a
sestavení bloků parcel a budov z načtených dat.

% Upřesněný výsledek
%% ML: prvni a druha veta rika to same, prepiste
%% LK: prvni vetu dam pryc
Podařilo se vytvořit takovou knihovnu, která
splňuje výše zmíněné zadání.
%% ML: druha veta opakuje to co jiz bylo receno, preformulujte
%% LK: preformulovano
%% ML: neuplne -> verejne dostupne (tj. data bez bloku NEMO - pokud si dobre vzpominam)
%% LK: opraveno
Veřejně dostupná data neobsahují datovou skupinu Nemovitosti, ve které
jsou obsaženy bloky parcel a budov. Proto knihovna oba bloky sestavuje
%% ML: proc je to nutne? -> vizualizace dat v pluginu, nevim, zda to
%% mate nekde jasne receno, minimalne, zde by to melo byt ucineno
%% LK: dopsano
%% ML: lepsi
a ukládá je do databáze, která vznikne při načtení dat VFK driverem
(viz kap. \ref{subsec:gdal_vfk}) knihovny GDAL. Sestavení obou bloků
má pro funkčnost zásuvného modulu zásadní vliv, poněvadž jsou v
mapovém okně QGISu vizualizovány.

% Komplikace
%% ML: nektere linky jsem odstranil, opakuji se
Při vytváření knihovny se objevily komplikace, které tvorbu práce
zpomalily. Verze knihovny GDAL 2.1.3 nesestavuje geometrii automaticky
přímo při načítání dat, ale až po provedení dotazu na konkrétní
geometrii. Dále VFK Driver do vzniklé databáze po načtení dat
nepřidává tabulky s geometrií a souřadnicovým systémem, tudíž vzniklá
%% ML: SQLite driver
%% LK: prepsano
databáze není prostorová a SQLite driver nedokáže rozeznat typ geometrie datových
%% ML: hlavne typ geometrie (bloky jako takove ano)
%% LK: opraveno
bloků. Tabulky byly v databázi vytvořeny a SQL driver poté dokázal typ
geometrie bloků přečíst. V nejnovější verzi knihovny GDAL jsou oba
nedostatky odstraněny. Rozšíření zásuvného modulu bylo vyvíjeno na
%% ML: veta o windows a vyvoji je zcela subjektivni, pokud umite
%% pouzivat vyvojove prostredky pod Windows, je to asi bajecny system
%% ;-)
%% ML: jde hlavne o to, ze byl testovan jak pod Linuxem a Windows, to
%% je podstatne
%% LK: veta o windows smazana
%% ML: v poradku
dvou operačních systémech -- Linux a Windows. Velkou komplikaci
způsobilo použití systémových proměnných, které v prostředí Linuxu
fungovalo a v prostředí Windows nikoliv. Situaci vyřešilo až
alternativní definování proměnné přes příkaz SetConfigOption().

%Výsledek
Výsledkem bakalářské práce je rozšířený zásuvný modul pro práci s daty
katastru nemovitostí ve formátu \zk{VFK} pro volně dostupný
geografický informační systém QGIS. Přidané rozšíření umožňuje načít
nejen zpoplatněná, ale i veřejně dostupná data ve formátu \zk{VFK} a sestaví bloky
parcel a budov, které jsou v mapovém okně systému QGIS 
vizualizovány včetně parcelních čísel.

%Další vývoj
Funkčnost knihovny byla testována na datech z katastrálního území
Abertamy, které obsahuje 1680 parcel a 470 budov. Velikost \zk{VFK}
souboru je 6,7 MB. U objemnějších dat trvá sestavování geometrie
% ML: ve vete mate dvakrat sestaveni geometrie
% LK: odtraneno
%% ML: mirne upraveno
%% LK: diky
výrazně déle. Tato operace by se dala pravděpodobně zrychlit, kdyby
probíhala přímo v
%% ML: proc? - GDAL ovladace jsou napsany v programovacim jazyku C++, ale ma to vice duvodu.
%% LK: jak by tedy bylo mozne sestavovani zrychlit?
%% ML: bez vyvoje a testovanim tezko rici ;-)
%% LK: takze takhle napsane staci?
prostředí VFK driveru. Do navazujícího vývoje patří zprovoznění
vyhledávání podle parcelního čísla i ve veřejně poskytovaných
datech. Import dat je třeba přesunout do separátního vlákna, což
umožní zobrazení informací o průběhu sestavování parcel a budov.

%Dokumentace
Tato práce svým obsahem podrobně dokumentuje funkčnost a způsob vzniku
%% ML: a jeji integrace do pluginu
%% LK: opraveno
nově vytvořené knihovny včetně integrace do zásuvného modulu. V příloze jsou informace o knihovně ještě
rozšířeny o diagram popisující princip sestavení hranic
%% ML: posledni veta je prilis dlouha, zkuste preformulovat a zlepsit
%% stylistickou uroven
%% LK: opraveno
parcel \ref{fig:logika_geometrie}. Další přílohou je návod jak načítat
data do zásuvného modulu (\ref{sec:nacteni_dat_ukazka}). Poslední
přílohou je návod na stažení veřejně poskytovaných neúplných dat
výměnného formátu katastru (\ref{sec:stazeni_dat_ukazka}). Stažení
těchto dat je potřebné pro využití nového rozšíření.

%Zdrojové kódy
%% ML: ve vasem gitu bakalarky (src)
%% LK: doplneno
%% ML: doplneno zalomeni radku
Zdrojové kódy knihovny jsou ke stažení ve složce \textit{src} na adrese:\newline \href{https://github.com/ctu-geoforall-lab-projects/bp-kettner-2018}{https://github.com/ctu-geoforall-lab-projects/bp-kettner-2018}.

%Distribuce
%Jaké problémy se objevily-operační systémy, jak se naplnila očekávání.
%Shrnout cíl práce a popsat výsledek-co knihovna/zásuvný modul nyní umí.
%Možnost dalšího vylepšení.
%% ML: doufam, ze bude kod zaclenen do repositare pluginu, pote jiz
%% bude distribuce standardni
%% LK: doplneno
%% ML: preformulovano
%% LK: dekuji
V dlouhodobém plánu je začlenění výsledku práce do zdrojových kódů
zásuvného modulu a umožnění otestovat novou funkcionalitu běžnými
uživateli programu QGIS.
