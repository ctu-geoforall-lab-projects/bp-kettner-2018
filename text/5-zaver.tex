\chapter{Závěr}
\label{5-zaver}
%Shrnutí cíle práce
Cílem mojí bakalářské práce bylo vytvoření nové knihovny psané v
jazyce Python, která měla rozšířit funkcionalitu již existujícího
zásuvného modulu pro aplikaci QGIS usnadňujícího práci s českými
katastrálními daty ve formátu \zk{VFK} o načítání neúplných veřejně
dostupných dat výměnného formátu katastru nemovitostí a o sestavení
bloků parcel a budov z načtených dat.

%Upřesněný výsledek
Podařilo se vytvořit novou knihovnu, která rozšiřuje zásuvný
modul. Rozšíření vzniklou knihovnou umožňuje nově načítání i neúplných
dat výměnného formátu katastru nemovitostí. Knihovna dále sestavuje z
neúplných dat bloky parcel a budov a ukládá je do databáze, která
vznikne při načtení dat VFK Driverem[\ref{subsec:gdal_vfk}] knihovny
GDAL.

%Komplikace
Při vytváření knihovny se objevily komplikace, které tvorbu práce
zpomalily. Verze knihovny GDAL 2.1.3, ze které byl používán VFK
Driver[\ref{subsec:gdal_vfk}], nesestavuje geometrii automaticky přímo
při načítání dat, ale až po provedení dotazu na konkrétní
geometrii. Dále VFK Driver do vzniklé databáze po načtení dat
nepřidává tabulky s geometrií a souřadnicovým systémem, tudíž vzniklá
databáze není prostorová a SQL driver nedokáže rozeznat datové
bloky. Tabulky byly v databázi vytvořeny a SQL Driver poté dokázal
datové bloky přečíst. V nejnovější verzi knihovny GDAL jsou oba
nedostatky odstraněny. Rozšíření zásuvného modulu bylo vyvíjeno na
dvou operačních systémech -- Linux a Windows. Windows se několikrát
ukázal jako nevhodný operační systém pro vývoj. Velkou komplikaci
způsobilo použití systémových proměnných, které v prostředí Linuxu
fungovalo a v prostředí Windows nikoliv. Situaci vyřešilo až
alternativní definování proměnné přes příkaz SetConfigOption().

%Výsledek
Výsledkem bakalářské práce je rozšířený zásuvný modul pro práci s daty
katastru nemovitostí ve formátu \zk{VFK} pro volně dostupný
geografický informační systém QGIS. Přidané rozšíření umožňuje načít
nejen úplná, ale i neúplná data ve formátu \zk{VFK} a sestaví bloky
parcel a budov, které jsou v mapovém okně systému QGIS vizualizovány
včetně parcelních čísel.

%Další vývoj
Funkčnost knihovny byla testována na datech z katastrálního území
Abertamy, které obsahuje 1680 parcel a 470 budov. Velikost \zk{VFK}
souboru je 6,7 MB. U objemnějších dat trvá sestavování geometrie
výrazně déle. Sestavování geometrií by se dalo zrychlit, kdyby
sestavování geometrie hranic parcel a budov probíhalo přímo v
prostředí VFK Driveru.

%%LK:Je
                                                                                                                                                                                                                                                                                                                                                           %%to
                                                                                                                                                                                                                                                                                                                                                           %%pravda?
                                                                                                                                                                                                                                                                                                                                                           %%Nejsem
                                                                                                                                                                                                                                                                                                                                                           %%si
                                                                                                                                                                                                                                                                                                                                                           %%jistý


Tato práce svým obsahem podrobně dokumentuje funkčnost a způsob vzniku
nově vytvořené knihovny. V příloha práce jsou informace o knihovně
ještě rozšířeny o diagram popisující princip sestavení hranic
parcel[\ref{fig:logika_geometrie}]. Je ukázáno krok za krokem jak
načítat do zásuvného modulu data[\ref{sec:nacteni_dat_ukazka}] a jak
si správně stáhnout veřejně poskytovaná neúplná data výměnného formátu
katastru[\ref{sec:stazeni_dat_ukazka}], které jsou zdrojem pro nové
rozšíření.

\textbf{!Kde budou zdrojové kódy ke stažení!}

%Jaké problémy se objevily-operační systémy, jak se naplnila očekávání.
%Shrnout cíl práce a popsat výsledek-co knihovna/zásuvný modul nyní umí.
%Možnost dalšího vylepšení.
(Distribuce zásuvného modulu)?
