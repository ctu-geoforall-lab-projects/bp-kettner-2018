\chapter{Závěr}
\label{5-zaver}
%Shrnutí cíle práce

%% ML: vyskrtnete ``moji'', jasne veci neni treba opakovat
%% ML: odstavec - jedno ohromne souveti, prepiste...
Cílem mojí bakalářské práce bylo vytvoření nové knihovny psané v
jazyce Python, která měla rozšířit funkcionalitu již existujícího
zásuvného modulu pro aplikaci QGIS usnadňujícího práci s českými
katastrálními daty ve formátu \zk{VFK} o načítání neúplných veřejně
dostupných dat výměnného formátu katastru nemovitostí a o sestavení
bloků parcel a budov z načtených dat.

%Upřesněný výsledek
Podařilo se vytvořit novou knihovnu, která rozšiřuje zásuvný
%% ML: druha veta opakokuje to co jiz bylo receno, preformulujte
%% ML: neuplne -> verejne dostupne (tj. data bez bloku NEMO - pokud si dobre vzpominam)
modul. Rozšíření vzniklou knihovnou umožňuje nově načítání i neúplných
dat výměnného formátu katastru nemovitostí. Knihovna dále sestavuje z
%% ML: proc je to nutne? -> vizualizace dat v pluginu, nevim, zda to
%% mate nekde jasne receno, minimalne, zde by to melo byt ucineno
neúplných dat bloky parcel a budov a ukládá je do databáze, která
vznikne při načtení dat VFK driverem (viz kap. \ref{subsec:gdal_vfk})
knihovny GDAL.

%Komplikace
Při vytváření knihovny se objevily komplikace, které tvorbu práce
zpomalily. Verze knihovny GDAL 2.1.3 nesestavuje geometrii automaticky
přímo při načítání dat, ale až po provedení dotazu na konkrétní
geometrii. Dále VFK Driver do vzniklé databáze po načtení dat
nepřidává tabulky s geometrií a souřadnicovým systémem, tudíž vzniklá
%% ML: SQLite driver
databáze není prostorová a SQL driver nedokáže rozeznat datové
%% ML: hlavne typ geometrie (bloky jako takove ano)
bloky. Tabulky byly v databázi vytvořeny a SQL driver poté dokázal
datové bloky přečíst. V nejnovější verzi knihovny GDAL jsou oba
nedostatky odstraněny. Rozšíření zásuvného modulu bylo vyvíjeno na
%% ML: veta o windows a vyvoji je zcela subjektivni, pokud umite
%% pouzivat vyvojove prostredky pod Windows, je to asi bajecny system
%% ;-)
%% ML: jde hlavne o to, ze byl testovan jak pod Linuxem a Windows, to
%% je podstatne
dvou operačních systémech -- Linux a Windows. Windows se několikrát
ukázal jako nevhodný operační systém pro vývoj. Velkou komplikaci
způsobilo použití systémových proměnných, které v prostředí Linuxu
fungovalo a v prostředí Windows nikoliv. Situaci vyřešilo až
alternativní definování proměnné přes příkaz SetConfigOption().

%Výsledek
Výsledkem bakalářské práce je rozšířený zásuvný modul pro práci s daty
katastru nemovitostí ve formátu \zk{VFK} pro volně dostupný
geografický informační systém QGIS. Přidané rozšíření umožňuje načít
nejen úplná, ale i neúplná data ve formátu \zk{VFK} a sestaví bloky
parcel a budov, které jsou v mapovém okně systému QGIS vizualizovány
včetně parcelních čísel.

%Další vývoj
Funkčnost knihovny byla testována na datech z katastrálního území
Abertamy, které obsahuje 1680 parcel a 470 budov. Velikost \zk{VFK}
souboru je 6,7 MB. U objemnějších dat trvá sestavování geometrie
% ML: ve vete mate dvakrat sestaveni geometrie
výrazně déle. Sestavování geometrií by se dalo zrychlit, kdyby
sestavování geometrie hranic parcel a budov probíhalo přímo v
%% ML: proc? - GDAL ovladace jsou napsany v programovacim jazyku C++, ale ma to vice duvodu.
prostředí VFK driveru.

%%LK:Je
                                                                                                                                                                                                                                                                                                                                                           %%to
                                                                                                                                                                                                                                                                                                                                                           %%pravda?
                                                                                                                                                                                                                                                                                                                                                           %%Nejsem
                                                                                                                                                                                                                                                                                                                                                           %%si
                                                                                                                                                                                                                                                                                                                                                           %%jistý


Tato práce svým obsahem podrobně dokumentuje funkčnost a způsob vzniku
%% ML: a jeji integrace do pluginu
nově vytvořené knihovny. V příloze jsou informace o knihovně ještě
rozšířeny o diagram popisující princip sestavení hranic
%% ML: posledni veta je prilis dlouha, zkuste preformulovat a zlepsit
%% stylistickou uroven
parcel \ref{fig:logika_geometrie}. Je ukázáno krok za krokem jak
načítat do zásuvného modulu data (\ref{sec:nacteni_dat_ukazka}) a jak
si správně stáhnout veřejně poskytovaná neúplná data výměnného formátu
katastru (\ref{sec:stazeni_dat_ukazka}), které jsou zdrojem pro nové
rozšíření.

%% ML: ve vasem gitu bakalarky (src)
\textbf{!Kde budou zdrojové kódy ke stažení!}

%Jaké problémy se objevily-operační systémy, jak se naplnila očekávání.
%Shrnout cíl práce a popsat výsledek-co knihovna/zásuvný modul nyní umí.
%Možnost dalšího vylepšení.
%% ML: doufam, ze bude kod zaclenen do repositare pluginu, pote jiz
%% bude distribuce standardni
(Distribuce zásuvného modulu)?
