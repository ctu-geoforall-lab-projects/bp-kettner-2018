%%%%%%%%%%%%%%%%%%%%%%%%%%%%%%%%%%%%%%%%%%%%%%%%%%%%%%%%%%%%%%%%%%%%%%%%%%%%%%%%%%%%%%%%%%%%%%%%%%%%%%%
%%													%%
%% 	BAKALÁŘSKÁ PRÁCE -  Rozšíření zásuvného modulu QGIS pro práci s katastrálními daty
%%                       o podporu veřejně dostupných dat ve formátu VFK			%%
%% 				 Lukáš Kettner						%%
%%													%%
%% pro formátování využita šablona: http://geo3.fsv.cvut.cz/kurzy/mod/resource/view.php?id=775 	%%
%%													%%
%%%%%%%%%%%%%%%%%%%%%%%%%%%%%%%%%%%%%%%%%%%%%%%%%%%%%%%%%%%%%%%%%%%%%%%%%%%%%%%%%%%%%%%%%%%%%%%%%%%%%%% 

\documentclass[%
  12pt,         			% Velikost základního písma je 12 bodů
  a4paper,      			% Formát papíru je A4
  oneside,       			% Oboustranný tisk
  pdftex,				    % překlad bude proveden programem 'pdftex' do PDF
%%%  draft
]{report}       			% Dokument třídy 'zpráva'
%

\newcommand{\Fbox}[1]{\fbox{\strut#1}}

\usepackage[czech, english]{babel}	% použití češtiny, angličtiny
\usepackage[utf8]{inputenc}		% Kódování zdrojových souborů je UTF8

\usepackage[square,sort,comma,numbers]{natbib}

\usepackage{caption}
\usepackage{subcaption}
\captionsetup{font=small}
\usepackage{enumitem} 
\setlist{leftmargin=*} % bez odsazení

\makeatletter
\setlength{\@fptop}{0pt}
\setlength{\@fpbot}{0pt plus 1fil}
\makeatletter

\usepackage[dvips]{graphics}
\usepackage{graphicx}   
\usepackage{color}
\usepackage{transparent}
\usepackage{wrapfig}
\usepackage{float} 

\usepackage{cmap}           
\usepackage[T1]{fontenc}    

\usepackage{textcomp}
\usepackage[compact]{titlesec}
\usepackage{amsmath}
\addtolength{\jot}{1em} 

\usepackage{chngcntr}
\counterwithout{footnote}{chapter}

\usepackage{acronym}
\usepackage{multirow} %pouziti u tabulky, vice radku v jednom
\usepackage[
    unicode,                
    breaklinks=true,        
    hypertexnames=false,
    colorlinks=true, % true for print version
    citecolor=black,
    filecolor=black,
    linkcolor=black,
    urlcolor=black
]{hyperref}         

\usepackage{url}
\usepackage[export]{adjustbox} %pro vycentrovani zadani.jpg
\usepackage{fancyhdr}
%\usepackage{algorithmic}
\usepackage{algorithm}
\usepackage{listings}
\definecolor{backcolour}{rgb}{0.95,0.95,0.92}
\definecolor{codegreen}{rgb}{0,0.6,0}
\lstdefinestyle{mystyle}{
	backgroundcolor=\color{backcolour},
	rulecolor=\color{backcolour},
	captionpos=b,
	numbers=left,
	numbersep=5pt,
	stringstyle=\color{codegreen},
	frame=shadowbox
}
\lstset{style=mystyle}
\usepackage{algcompatible}
\renewcommand{\ALG@name}{Pseudokód}% Update algorithm name
\def\ALG@name{Pseudokód}

\usepackage[
  cvutstyle,          
  bachelor           
]{thesiscvut}


\newif\ifweb
\ifx\ifHtml\undefined % Mimo HTML.
    \webfalse
\else % V HTML.
    \webtrue
\fi 

\renewcommand{\figurename}{Obrázek}
\def\figurename{Obrázek}

%%%%%%%%%%%%%%%%%%%%%%%%%%%%%%%%%%%%%%%%%%%%%%%%%%%%%%%%%%%%%%%%%
%%%%%%%%%%% Definice informací o dokumentu  %%%%%%%%%%%%%%%%%%%%%
%%%%%%%%%%%%%%%%%%%%%%%%%%%%%%%%%%%%%%%%%%%%%%%%%%%%%%%%%%%%%%%%%

%% Název práce
\nazev{Rozšíření zásuvného modulu QGIS pro práci s~katastrálními daty o podporu veřejně dostupných dat ve formátu VFK}{}

%% Jméno a příjmení autora
\autor{Lukáš}{Kettner}

%% Jméno a příjmení vedoucího práce včetně titulů
\garant{Ing.~Martin~Landa,~Ph.D.}

%% Označení programu studia
\programstudia{Geodézie a~kartografie}{}

%% Označení oboru studia
\oborstudia{Geodézie, kartografie a~geoinformatika}{}

%% Označení ústavu
\ustav{Katedra geomatiky}{}

%% Rok obhajoby
\rok{2018}

%Mesic obhajoby
\mesic{únor}

%% Místo obhajoby
\misto{Praha}

%% Abstrakt
\abstrakt{Tato bakalářská práce je zaměřena na rozšíření již
  existujícího softwarového nástroje pro práci s katastrálními daty o
  možnost využití nekomerčních, volně dostupných (nezpoplatněných)
  %% ML: dvakrat v jedne vete: sestaveni a sestavit
  %% LK: souvětí změněno v jednu větu
  dat ve formátu VFK. Konkrétně se jedná o sestavení bloků PAR a BUD
  (parcel a budov) z obsažených geometrických a popisných
  informací. Zásuvný modul bude vyvíjen pro prostředí open
  source nástroje QGIS v programovacím jazyce Python.}  {This bachelor
  %% ML: anglicky text je nutny revidovat, obsahuje preklepy,
  %% stylisticke a gramaticke chyby
  %% LK: opraveno
  thesis is focused on development of already existing plugin for
  working with cadastral data in VFK format. Final outcome will enable users to operate with non-commercial data in VFK
  format provided free of charge. The plugin creates new blocks PAR and BUD (parcels and
  buildings) from geometric and description information which are
  included. The plugin will be developed in open source project QGIS
  in programming language Python.}

%% Klíčová slova
\klicovaslova
{QGIS, zásuvný~modul, Python, GDAL, VFK}
{QGIS, plugin, Python, GDAL, VFK}

%%%%%%%%%%%%%%%%%%%%%%%%%%%%%%%%%%%%%%%%%%%%%%%%%%%%%%%%%%%%%%%%%%%%%%%%

%%%%%%%%%%%%%%%%%%%%%%%%%%%%%%%%%%%%%%%%%%%%%%%%%%%%%%%%%%%%%%%%%%%%%%%%
%% Nastavení polí ve Vlastnostech dokumentu PDF
%%%%%%%%%%%%%%%%%%%%%%%%%%%%%%%%%%%%%%%%%%%%%%%%%%%%%%%%%%%%%%%%%%%%%%%%
\nastavenipdf
%%%%%%%%%%%%%%%%%%%%%%%%%%%%%%%%%%%%%%%%%%%%%%%%%%%%%%%%%%%%%%%%%%%%%%%

%%% Začátek dokumentu
\begin{document}

\catcode`\-=12  % pro vypnuti aktivniho znaku '-' pouzivaneho napr. v \cline 

% aktivace záhlaví
\zahlavi

% předefinování vzhledu záhlaví
\renewcommand{\chaptermark}[1]{%
	\markboth{\MakeUppercase
	{%
	\thechapter.%
	\ #1}}{}}

% Vysázení přebalu práce
%\vytvorobalku

% Vysázení titulní stránky práce
\vytvortitulku

% Vysázení listu zadani
\stranka{}%
	{\includegraphics[scale=1, center]{./pictures/zadani.jpg}}%\sffamily\Huge\centering\ }
%ZDE VLOŽIT LIST ZADÁNÍ}%
	%{\sffamily\centering Z~důvodu správného číslování stránek}

% Vysázení stránky s abstraktem
\vytvorabstrakt

% Vysázení prohlaseni o samostatnosti
\vytvorprohlaseni

% Vysázení poděkování
\stranka{%nahore
       }{%uprostred
       }{%dole
       \sffamily
	\begin{flushleft}
		\large
		\MakeUppercase{Poděkování}
	\end{flushleft}
	\vspace{1em}
		%\noindent
	\par\hspace{2ex}
	{Na tomto místě bych chtěl poděkovat především vedoucímu práce, Ing. Martinu Landovi, PhD., nejen za cenné rady a připomínky k zlepšení práce po všech stránkách, ale také za velké množství věnovaného času při objasňování problematiky. Dále chci poděkovat také svým blízkým, kteří mi v případě potřeby byli vždy nápomocni.}
}

% Vysázení obsahu
\obsah

% Vysázení seznamu obrázků
\seznamobrazku

% Vysázení seznamu tabulek
\seznamtabulek

% jednotlivé kapitoly
\chapter{Úvod}
\label{1-uvod}
%Úvod úvodu

%% ML: prvni veta (vybrat poradne) je naprosto nevhodna, jde o
%% vedeckou praci, nejste v hospode ;-), nutne prepsat ML: cely prvni
%% odstavec zni kostrbate, text se Vam rozpada po stylisticke i
%% logicke strance, zkuste cely odstravec prepsat
Téma bakalářské práce jsem si chtěl vybrat pořádně. Zvolit jsem si
chtěl takové téma, které mě zaujme, něco nového se naučím a výsledek
bude ideálně dál využitelný. Proto bylo jasné, že o orientačním běhu,
který je mojí srdeční záležitostí, psát nebudu, poněvadž tím bych moc
nového neobjevil. Výběr tématu jsem začal řešit už v souvislosti s
%% ML: osobni pribeh vynechte, pro volbu tematu nemel az tak zasadni
%% vlic
možným výjezdem do zahraničí, kam jsem chtěl během zimního semestru
vyrazit. To se po návštěvě studijního oddělení ukázalo jako značně
komplikované a proto jsem se rozhodl zůstat s prací na domácí fakultě.

%Proces výběru
Během bakalářského studia mě bavilo programování výpočetních skriptů a
%% ML: na ktery zni kostrbate, zkuste celou vetu prepsat
taky se mi zalíbil geografický informační systém, na který jsme měli
ve třetím ročníku samostatný předmět. Proto jsem s bakalářskou prací
zamířil na katedru geomatiky. První téma, které jsem dostal na
vyzkoušení, byla tvorba zásuvného modulu pro práci s registrem územní
identifikace, adres a nemovitostí. Projekt se jmenoval
\textit{qgis-ruain-plugin} a poprvé jsem si zde vyzkoušel práci s
programovacím jazykem Python. Za úkol jsem měl osahat si prostředí
%% ML: co znamena ``trosku''? Musite byt konkretni a vecny, preci
%% jenom jde o zaverecnou praci technicke povahy, nejde o slohove
%% cviceni
zásuvného modulu a trošku změnit zdrojový kód. Během testování jsem si
samostudiem osvojil základy programovacího jazyka.

%Budoucnost
%% ML: Prvni vetu zkuste prepsat (ubirat svoji)
Práce mě zaujala natolik, že jsem se rozhodl ubírat svojí bakalářskou
práci směrem zásuvného modulu pro geografický informační systém
QGIS. Lákalo mě vyzkoušet si naprogramovat něco užitečného a
funkčního. Zároveň bych si chtěl během tvorby bakalářské práce udělat
%% ML: mozna prilis osobni
jasnější obraz o tom, jakým směrem by se mohla moje studia
ubírat. Stále nemám jasnou volbu mezi geomatikou a inženýrskou
geodézií. Něco mě táhne k geomatice, tak doufám, že mi to bakalářská
práce potvrdí.

%Motivace
%% ML: tady osobni nadech jeste eskaluje, zkuste prepsat ML: uvod
%% pisete v budoucim case, to neni uplne vhodne, text vznika behem
%% prace a nikoliv v minulosti
Začátek nebude jednoduchý, s tím do tvorby práce jdu. Očekávám, že
programování nové knihovny, která je základním kamenem rozšíření
funkcionality zásuvného modulu, bude v hodně věcech podobné psaní
výpočetních skriptů.  Věřím, že zúročím zkušenosti jak z výpočetních
skriptů, tak ze zkušební práce na zásuvném modulu a práce mi půjde
dobře od ruky.

%Představení zadaní
Jako téma práce jsem si zvolil \textit{Rozšíření zásuvného modulu QGIS
  pro práci s katastrálními daty o podporu veřejně dostupných dat ve
  formátu VFK}.  Moje práce bude dál rozšiřovat funkčnost zásuvného
modulu z diplomové práce Bc. Štěpána Bambuly \textit{Rozšíření
  nástroje pro práci s katastrálními daty v programu QGIS}. Práce pana
Bambuly navazovala na již existující nástroj a rozšířila ho o
zpracování a vizualizaci datových vět změnových souborů \zk{VFK}. Dále
nástroj přepsal do jazyka Python, aby usnadnil distribuci zásuvného
modulu v prostředí geografického informačního systému QGIS.

%Cíl práce
%% ML: zvazte prepsani celeho textu do pritomneho casu (Cilem teto
%% prace je...)
Cílem mojí práce bude tento zásuvný modul v jazyce Python ještě
rozšířit o novou knihovnu, která umožní nahrání, prohlížení a
%% ML: spojka 'a' evokuje pocit, ze jde o dve formy, data mohou byt
%% 'neuplna' a take 'verejna': data verejne neobsahuji vsechny datove
%% bloky v porovnani s neverejnymi daty, ktera jsou poskytovana za
%% uplatu
vyhledávání i v neúplných a veřejně dostupných datech ve formátu
\zk{VFK}. Rozšíření bude zaměřené na sestavení bloků PAR a BUD (parcel
a budov), které v těchto datech nejsou obsaženy, ale mohou být z
%% ML: nemluvil bych o 'geometrickych blocich' ale datovych blocich s
%% geometrickou sloznou popisu
přítomných dat sestaveny. Jedná se o dva nejdůležitější geometrické
bloky, které jsou nezbytné pro vizualizaci dat a tvoří datový blok
nemovitostí. Po vytvoření knihovny by měla následovat integrace do již
existujícího zásuvného modulu. Budu se maximálně snažit využít veškeré
v datech obsažené informace, aby se výsledek co nejvíce podobal datům
úplným. Tím dojde i k zachování většiny funkcí rozšiřovaného zásuvného
modulu.

%Struktura práce
%% ML: po presani do pritomneho casu, muzete pridat odkazy na
%% jednotlive kapitoly
Samotná práce bude logicky uspořádána do dvou celků. Prvním bude
teoretická část zabývající se představením informačního systému
katastru nemovitostí(\zk{ISKN}) a jeho historie. Dále bude rozebrána
základní struktura výměnného formátu katastru (\zk{VFK}), ve kterém
jsou data poskytována. Přidám porovnání úplného a neúplného formátu
doplněné o přehled datových bloků, pro které je nutné sestavit
geometrii, aby šlo bloky PAR a BUD knihovnou také sestavit. Představím
použitou technologii, kam patří například programovací jazyk Python a
knihovnu GDAL.

Druhá část už bude zaměřená čistě na praktickou stránku
práce. Podrobně představím funkcionalitu jednotlivých tříd a členských
metod z nově vzniklé knihovny a také funkčnost knihovny
samotné. Součástí praktické části bude i integrace vzniklé knihovny do
již existujícího zásuvného modulu, která bude také představena a
doplněna o ukázky načítání dat.

%%%POZNAMKY
%-Data výměnného formátu katastru(\zk{VFK}) jsou poskytována ve dvou podobách.

%Rešerše% \textbf{Rešerše:} 
%Odkaz v textu% \footnote{\url{http://theses.cz/id/o3vhp8/Diplomov_prce_Lokov.pdf}}
%Kurzíva% \textit{}
%Zkratka% \zk{ZHN}
%\footnote{}

\chapter{Teoretický základ}
\label{2-teorie}
%co v kapitole uvedu?
V této kapitole představím Informační systém katastru nemovitostí a doplním o způsob poskytování dat z katastru nemovitostí. Text této kapitoly vychází z informací o \zk{ISKN} na webové stránce cuzk.cz. \cite{iskn}

\section{Informační systém katastru nemovitostí}
Katastr nemovitostí se řadí mezi datově nejrozsáhlejší informační systémy státní správy. Pro výkon státní správy a zajištění uživatelských služeb byl v letech 1997-2001 zřízen Informační systém katastru nemovitostí(\zk{ISKN}), který sjednotil vedení a správu katastru nemovitostí do jediného informačního systému. Aktuální data z katastru nemovitostí jsou dostupná přes službu Dálkový přístup na síti internetu po registraci během několika minut. 
%zdroj: http://www.cuzk.cz/Katastr-nemovitosti/O-katastru-nemovitosti/Informacni-system-katastru-nemovitosti-ISKN.aspx
\subsection{Vývoj ISKN}
\zk{ISKN} vznikl v letech 1997-2001 ve spolupráci s firmou APP Czech s.r.o.(NESS Czech s.r.o.). V roce 1998 došlo k dokončení digitalizace souboru popisných informací a nyní se digitalizuje souboj geodetických informací. Na všech katastrálních pracovištích byl \zk{ISKN} zprovozněn v roce 2001. Během následujících let byl systém průběžně laděn. V letech 2007 až 2010 došlo k centralizaci informačního systému do jediné databáze, čímž odpadlo replikování ze 107 lokálních databází a zrychlila se aktualizace dat v Dálkovém přístupu. Velkou výhodou \zk{ISKN} je možnost zavedení automatických kontrol při zápisu do katastru nemovitostí. Kvůli zvýšení bezpečnosti, na kterou byl při tvorbě systému kladen velký důraz, je celá infrastruktura zdvojena. Vzniklo tak primární a záložní centrum, které v případě výpadku primárního centra udrží \zk{ISKN} v provozu.
\subsection{Poskytování dat}
Český úřad zeměměřický a katastrální( \zk{ČUZK}) poskytuje široké spektrum data v papírové i digitální podobě. Pro tuto práci je podstatný výstup dat ve výměnném formátu \zk{ISKN} v textovém tvaru, který obsahuje popisné i grafické informace dle zadané kombinace bloků(\textit{viz Tab. 2.1. Kombinace datových bloků}). V digitální podobě jsou \zk{ČUZK} poskytována nejen data ve formátu \zk{VFK} ale také ve formátu shapefile(SHP) přes službu Dálkový přístup. %http://www.gisoft.cz/Moduly/ImportVFK
\begin{table}[h!] %specifikace umisteni objektu-tabulky, ! trvá na umístění h-here
			\centering
			\caption{Kombinace datových bloků(zdroj:
\href{http://www.cuzk.cz/Katastr-nemovitosti/Poskytovani-udaju-z-KN/Vymenny-format-KN/Vymenny-format-NVF.aspx}{cuzk.cz})}
			\label{tab:komb_dat_skup}
			\begin{tabular}{|l|l|}
				\hline
				\textbf{Blok}           	& \textbf{Popis bloku}  	\\ \hline
				1. Nemovitosti				& parcely a budovy	\\ \hline
				2. Jednotky					& bytové jednotky	 \\ \hline
				3. Bonitní díly parcel      & kódy \zk{BPEJ} k parcelám              \\ \hline
				4. Vlastnictví             	& listy vlastnictví, oprávněné subjekty a vlastnické vztahy		 \\ \hline
				5. Jiné právní vztahy 		& ostatní právní vztahy kromě vlastnictví \\ \hline
				6. Řízení       			& údaje o řízení (vklad, záznam,…) a listiny          \\ \hline
				7. Prvky katastrální mapy 	& katastrální mapy v digitální podobě	 \\ \hline
				8. \zk{BPEJ}				& hranice \zk{BPEJ} včetně kódů	 \\ \hline
				9. Geometrický plán			& geometrické plány	 \\ \hline
				10. Rezervovaná čísla		& rezervovaná parcelní čísla a čísla \zk{PBPP}	 \\ \hline
				11. Definiční body 			& definiční body parcel a staveb	 \\ \hline
				12. Adresní místa 			& adresní místa budov	 \\ \hline
			\end{tabular}
		\end{table}
%tabulka zdroj: http://www.cuzk.cz/Katastr-nemovitosti/Poskytovani-udaju-z-KN/Vymenny-format-KN/Vymenny-format-NVF.aspx
Poskytování veškerých dat se řídí vyhláškou číslo 358/2013 Sb., o poskytování údajů z katastru nemovitostí.
\section{Výměnný formát katastru nemovitostí}
Obsah této kapitoly vychází z informací na stránkách \zk{ČUZK} o Výměnném formátu katastru nemovitostí a z dokumentu \textit{Struktura výměnného formátu informačního systému katastru nemovitostí České republiky} ze dne 7.11.2014.\cite{struktura_ISKN}
\subsection{Historie a vývoj}
Výměnný formát před \zk{ISKN} byl označován jako \textit{starý výměnný formát(SVF)} a obsahoval tři samostatné a oddělené části:
\begin{itemize}
	\item \textbf{Soubor popisných informací(SPI)} - informace o parcelách, vlastnících, nabývacích titulech
	\item \textbf{Soubor geodetických informací (SGI)} - informace o poloze nemovitosti
	\item \textbf{Digitální katastrální mapu (DKM)} - soubory ve formátu VKM
\end{itemize}
%zdroj: http://www.cuzk.cz/Katastr-nemovitosti/Poskytovani-udaju-z-KN/Vymenny-format-KN/Vymenny-format-KN-pred-ISKN.aspx
Podpora starého výměnného formátu skončila se vznikem \zk{ISKN}, protože v něm jsou data popisná a geodetická uložena ve společném datovém modulu. Proto byl vytvořen a postupně implementován \textit{nový výměnný formát(NVF)}. Jeho data jsou poskytována ve dvou časových režimech:
\begin{itemize}
\item \textbf{Prvotní data}\begin{itemize}
								\item Kompletní data pro konkrétní časové období
						   \end{itemize}

\item \textbf{Změny} \begin{itemize}
								\item Data obsahující pouze změny za konkrétní časové období. Lze zadávat datum od-do včetně času.
							\end{itemize}
\end{itemize}
Tento nový datový formát obsahuje dle požadované kombinace bloků popisnou i grafickou informaci včetně dat o řízení. Rozsah poskytovaných dat je možné definovat podle:
\begin{itemize}
		\item Územní jednotka (katastrální území, obec, okres, Česká republika)
		\item oprávněný subjekt
		\item výběr parcel
		\item výběr parcel polygonem v mapě
\end{itemize}
%http://www.cuzk.cz/Katastr-nemovitosti/Poskytovani-udaju-z-KN/Vymenny-format-KN/Vymenny-format-ISKN-v-textovem-tvaru.aspx
%http://geo.fsv.cvut.cz/~landa/publications/2005/diploma_thesis/martin.landa-thesis.pdf
\subsection{Struktura výměnného formátu ISKN}
V této kapitole představím nejzákladnější strukturu nového výměnného formátu \zk{ISKN}, velice podrobný popis je k dispozici v dokumentaci o struktuře.\cite{struktura_ISKN}

Výměnný formát je určený k vzájemnému předávání dat mezi systémem \zk{ISKN} a jinými systémy zpracování dat. Datový soubor výměnného formátu je textový soubor s kódováním češtiny\footnote{Pouze ve výjimečných případech lze poskytnout v kódování dle WIN1250} dle ČSN ISO 8859-2 (ISO Latin2) skládající se z:
\begin{itemize}
		\item hlavičky \verb|&H|
		\item datových bloků \verb|&B|
		\item koncového znaku \verb|&K|
\end{itemize}
Každý z datových bloků v sobě obsahuje informaci o atributech a jejich formátu následovanou vlastními datovými řádky. Každá věta hlavičky (\verb|&H|), definice bloku(\verb|&B|) i věta dat (\verb|&D|) je zakončena znaky <CR><LF>. %ukázku?
\subsection{Neúplná data výměnného formátu katastru}
\textbf{!Doplním na základě zjištěných informací od pána z úřadu.!} 
Pro bakalářskou práci jsou hlavní bezúplatně poskytovaná data z neharmonizovaných služeb \zk{ČUZK}. Bezplatně je možné získat pouze katastrální mapu, proto se jedná o data neúplná. Popisné informace data neobsahují. Data je možné stáhnout z internetu na adrese http://services.cuzk.cz/vfk, kde si zvolíme jestli chceme soubory s katastrální mapou po katastrálním území nebo katastrálních pracovištích.
%http://services.cuzk.cz/
%http://freegis.fsv.cvut.cz/gwiki/V%C3%BDm%C4%9Bnn%C3%BD_form%C3%A1t_ISKN
\section{Nástroje pro čtení dat VFK} 
K čtení dat informačního systému katastru(\zk{ISKN}) je k dispozici poměrně velké množství prostředků. Některé jsou komerční a jiné volně dostupné. V následující kapitole představím mnou dohledané softwary včetně v práci využitého VFK driveru. Pokud to bylo možné, načtení dat jsem si v jednotlivých softwarech vyzkoušel.
\subsection{GISOFT}
Jedná se o komerční modul od společnosti GISoft ve spolupráci se společností Bentley Systems pro MicroStation. Modul umožňuje import dat v novém i starém výměnném formátu katastru nemovitostí České republiky. Tento modul je k dispozici pro nadstaveby MGEO a SPIDER.\cite{gisoft}

\begin{figure}[H]
	 \centering
      \includegraphics[height=9cm]{./pictures/gisoft.png}
      \caption{Ukázka načtení VFK dat (zdroj:
\href{http://www.gisoft.cz/cze/files/Moduly/import-vfk.png}{gisoft.cz})}
      \label{fig:topol}
  \end{figure}
%zdroj: http://www.gisoft.cz/Moduly/ImportVFK
%obrazek: http://www.gisoft.cz/cze/files/Moduly/import-vfk.png
\subsection{Spirit VFK}
Samospustitelná desktopová aplikace určená pro převod dat katastru nemovitostí do libovolné geodatabáze ESRI. Do geodatabáze jsou importovány tabulky, relace a ostatní databázové objekty \zk{ISKN}. Výslednou databázi jde použít pro analytickou práci na datech kastartu nemovitostí nebo v aplikačních nadstavbách \textit{Spirit KN} a \textit{Spirit Portál} -- KN. Produkt je od společnosti GEOREAL s.r.o. a je zpoplatněný. Po krátké registraci je možné vyzkoušet 30-ti denní Trial verzi \footnote{http://mapy.georeal.cz/trialreg/}.\cite{spirit_vfk}
%zdroj: http://www.georeal.cz/cz/spirit-desktop/spirit-vfk
\subsection{TopoL xT}
TopoL xT je zpoplatněný software od společnosti TopoL. Je to obecný geografický systém určený pro přípravu, správu a analýzu geografických dat. Soubory výměnného formátu katastru nemovitostí - VFK jsou jedním z podporovaných formátu pomocí importu. Pro vyzkoušení je k dispozici demonstrační verze\footnote{http://www.topol.eu/files/download/topol/100/setup.exe}, ve které je omezený počet objektů, omezená velikost rastru a nelze tisknout v měřítku.\citep{topol}

\begin{figure}[H]
	 \centering
      \includegraphics[height=9cm]{./pictures/topol.png}
      \caption{Ukázka importu VFK dat v programu TopoL xT (zdroj:vlastní)}
      \label{fig:topol}
  \end{figure}
%http://www.topol.eu/articles/software#topol-xt
\subsection{Kokeš}
Interakční grafický systém Kokeš od firmy GEPRO s.r.o. je zaměřený na obor geodézie a na geoinformační systémy. V systému je možné řešit nejrůznější geodetické i konstrukční výpočty, vytvářet a aktualizovat kresby map, vést popisné údaje k objektům a bodům mapy, digitalizovat grafické podklady. Budování systému po základních a uživatelských funkcích umožňuje postupný vývoj a jednoduché ovládání. \cite{kokes_cvut}

Funkce import VFK umožňuje importovat jednotlivé soubory, kdy vstupem je jeden nebo více souborů VFK (nebo ZIP) a výsledkem jedna databáze SPI a výkresy katastrální mapy, orientační mapy, definičních bodů parcel a definičních bodů budov. %\citep{napoveda_kokes}

\begin{figure}[H]
	 \centering
      \includegraphics[width=10cm]{./pictures/kokes.png}
      \caption{Ukázka importu VFK dat v programu Kokeš (zdroj:vlastní)}
      \label{fig:kokes}
  \end{figure}
%zdroj: http://geo3.fsv.cvut.cz/~soukup/dip/bukovsky/1.htm, napoveda kokes k funkci import vfk
\subsection{ISKN Studio pro ArcGIS}
Software určený pro import dat formátu \zk{ISKN} do formátu geodatabáze. Pracuje s daty ve formátu NVF a umožňuje jejich zpracování do osobní, souborové a ArcSDE geodatabáze v MS SQL Server či Oracle. K ISKN Studiu jde možné doinstalovat odplněk ISKN View, který slouží k rychlému a jednoduchému vyhledávání v datech ISKN převedených pomocí softwaru ISKN Studio. Jedná se o zpoplatněný software. \cite{arcgis}

\begin{figure}[H]
	 \centering
      \includegraphics[height=9cm]{./pictures/iskn_studio.jpeg}
      \caption{Ukázka načtení VFK dat v ISKN Studiu (zdroj:
\href{https://www.arcdata.cz/uploads/media/general/0001/01/68f0bfd90cf19d903a57fc8457e1f228a7dd47f4.jpeg}{arcdata.cz})}
      \label{fig:iskn studio}
  \end{figure}
%zdroj:https://www.arcdata.cz/produkty/software-arcdata/import-iskn
%obrázek: https://www.arcdata.cz/uploads/media/general/0001/01/68f0bfd90cf19d903a57fc8457e1f228a7dd47f4.jpeg
\subsection{cadstudio}
V obou případech se jedná o konverzní aplikaci firmy CAD Studio určenou pro zpracování dat \zk{ISKN}.
\begin{itemize}
\item VFK2DB -- import dat do relační databáze Oracle nebo MS SQL Server, samostatný spustitelný program \cite{vfk2db} %http://www.cadstudio.cz/vfk
\item VFK2DWG -- automatický převod souboru či souborů VFK přímo na objekty AutoCADu a nimi svázané databázové tabulky, nadstavba AutoCADu \cite{vfk2dwg}

\begin{figure}[H]
	 \centering
      \includegraphics[width=14cm]{./pictures/vfk2dwg.png}
      \caption{Import VFK souborů (zdroj:
\href{http://www.cadstudio.cz/img/vfk2dwg11.gif}{cadstudio.cz})}
      \label{fig:iskn studio}
  \end{figure}
%http://www.cadstudio.cz/vfk2dwg
\end{itemize}
\subsection{knihovna GDAL} 
VFK Driver je součástí knihovny \zk{GDAL} a umožňuje čtení souborů výměnného formátu katastru(VFK). Driver, česky ovladač, slouží obecně k rozšíření funkcionality. Soubor \zk{VFK} je vnímán jako zdroj dat(\verb|OGR datasource|) s žádnou nebo více vrstvami(\verb|OGR layers|). Body jsou ve vrstvách reprezentovány jako \verb|wkbPoints|, linie a hranice jako \verb|wkbLineStrings| a plochy jako \verb|wkbPolygons|. VFK driver si během prvního čtení ukládá data do SQLite databáze, která se vytvoří ve stejném adresáři jako je vfk soubor. Opakované načtení je díky již vytvořené databázi výrazně rychlejší. Výhoda databáze je v snazším a rychlejším přístupu k datům. Dále si může uživatel pomocí proměnných \verb|OGR_VFK_DB_OVERWRITE| a \verb|OGR_VFK_DB_NAME| nastavit jestli bude vytvořená SQL databáze při načtení přepsaná(čtení stále z vfk souboru) a jaký bude název vytvořené databáze. Navíc je tento driver jako jediný z výše zmíněných nástrojů volně dostupný. \cite{vfk_driver}

\begin{figure}[H]
	 \centering
      \includegraphics[width=15cm]{./pictures/vfk_driver.png}
      \caption{Ukázka použití VFK driveru (zdroj:vlastní)}
      \label{fig:vfk driver}
  \end{figure}
%zdroj:http://gdal.org/drv_vfk.html
%http://freegis.fsv.cvut.cz/gwiki/VFK_/_GDAL
%http://freegis.fsv.cvut.cz/gwiki/V%C3%BDm%C4%9Bnn%C3%BD_form%C3%A1t_ISKN
%\subsubsection{SQLite driver}
%Ve verzi knihovny \zk{GDAL} 2.1.3 a nižší je nezbytné, aby databáze obsahovala tabulky s upřesněním podoby obsažené geometrie (\verb|geometry columns|) a souřadného systému (\verb|spatial_ref_sys|). Pokud databáze tabulky neobsahuje, SQLite driver nerozezná datový typ vrstvy. Novější verze knihovny si s absencí tabulek dokáže poradit, resp už si tabulky vytvoří sama.
%zdroj: http://gdal.org/drv_sqlite.html

%VFK/QGIS plugin
%http://freegis.fsv.cvut.cz/gwiki/VFK_/_QGIS_plugin
\section{Sestavení geometrie prvků}
K sestavení geometrie prvků dochází až když o ní uživatel požádá. Platí pro verzi knihovny GDAL 2.1 a nižší. Geometrie se sestavuje postupně po blocích dle schématu:
\begin{figure}[H]
	 \centering
      \includegraphics[width=10cm]{./pictures/Vfk-diagram-geom.png}
      \caption{Přehled datových bloků pro sestavení geometrie prvků digitální katastrální mapy (zdroj:
      \href{http://freegis.fsv.cvut.cz/wiki/images/thumb/8/8a/Vfk-diagram-geom.png/744px-Vfk-diagram-geom.png}{freegis.fsv.cvut.cz})}
      \label{fig:vfk diagram geom}
  \end{figure}
Pokud nás tedy zajímá geometrie bloku HP(hranic parcel), dojde k sestavení geometrie i pro všechny předchozí bloky - SOBR(souřadnice obrazu bodů polohopisu) a SBP(spojení bodů polohopisu). Obdobně je to při sestavování bloku OB(obrazy budov) - sestaví se geometrie i pro bloky SBP a SOBR. Zmíněné loky HP a OB jsou podstatné pro tuto práci, protože právě z nich budou pomocí knihovny sestaveny bloky PAR(parcely) a BUD(budovy).
%zdroj: http://freegis.fsv.cvut.cz/gwiki/VFK_/_GDAL
\chapter{Použité technologie}
\label{3-technologie}
%% ML: programy -> technologie (ci neco vice obecneho nez programy)
%% LK: -> technologie
V této kapitole budou zmíněny technologie, které byly pro tvorbu
bakalářské práce využity. Patří sem hlavně programovací jazyk Python,
%% ML: plugin vznika? zkuste preformulovat
%% LK: preformulovano jiz driv, jen pro poradek :-)
geografický informační systém QGIS, pro který je zásuvný modul vyvíjen.
%% ML: posledni vete vubec nerozumim, orientaci? Mate na mysli, to ze
%% jste pouzival Qt Creator pro tvorbu UI? Qt je graficky framework na
%% kterem je QGIS postaven.
%% LK: mel jsem na mysli Qt creator,zminim zde Qt
%% ML: OK
Pro práci důležitou technologií je i knihovna GDAL a grafický
framework\footnote{Softwarová struktura sloužící jako podpora při
  vývoji nových programů} Qt.

\section{Python}
\label{sec:python}
\begin{figure}[H]
	 \centering
      \includegraphics[width=5cm]{./pictures/python-logo.png}
      \caption{Logo Python (zdroj:
\href{https://www.python.org/static/community_logos/python-logo-master-v3-TM.png}{Python.org})}
      \label{fig:python}
  \end{figure}

%% ML: druhou cast souveti prepiste anebo uplne vynechte
%% LK: vynechána
%% ML: dobre
Za autora programovacího jazyka Python je považován Guido vam Roosum.
%% ML: drou vetu prepiste do cestiny, pocatek ... byl ... na
%% LK: prepsano
Počátek vývoje jazyka Python byl v roce 1990 na Stichtig Mathematisch
Centrum v Nizozemí. Hlavní princip
%% ML: osloveni Guido zni familiarne, odkud jste text cerpal?
%% LK: cerpano je z ucebnice jazyka Python, odkaz {ucebnicepython},
%% odkaz na zdroj presunut z konce textu pod tento odstavec
vychází z programovacího jazyka ABC. Python je volně dostupný včetně
standardních knihoven, dokumentace a zdrojových kódů. V roce 2001
vznikla nezisková organizace Python Software Foundation, která je
vlastníkem veškerých intelektuálních materiálů souvisejících s
programovacím jazykem
%% ML: poznamka pod carou: velmi zjednodusena definice open source
%% ML: lepsi
%% LK: ok
Python. Spravuje open source\footnote{Jedná se o technologie s volně
  poskytovanými zdrojovými kódy, na vývoji se tak může podílet
  kdokoliv} licence Pythonu od verze 2.1 a výš. Zároveň se stará o
%% ML: posledni veta (pod), ktera se na vyvoji jazyka take podilela a pod...
%% ML: zbytek vety jsem odstranil
%% LK: ok
ochranou známku jazyka. Jedním ze sponzorů neziskové organizace je
společnost Digital Creations. \cite{ucebnicepython}

Python účinně a efektivně pracuje s vysokoúrovňovými datovými
typy. Syntaxe jazyka a dynamické typy z něj dělají vhodný nástroj pro
%% ML: mate v seznamu zkratek?, zde nemate rozepsanu
%% LK: ano mam
%% ML: OK
psaní skriptů a rychlý vývoj aplikací (\zk{RAD}). Jazyk si snadno
%% ML: cteni syntaxe nezni cesky,..
%% LK: prepsano
oblíbí začátečníci, pro které je struktura jazyka snadno pochopitelná. 
%% ML: co znamena ``interpret''
%% LK: myslel jsem opreacni systemy
%% ML: v poradku
Další výhodou je spustitelnost na velkém množství operačních systémů
zahrnujících Linux, Windows, MacOS. \cite{python, diveintopython}

%obrazek: https://www.raspberrypi.org/documentation/usage/python/images/python-logo.png
%zdroje:https://docs.python.org/3/tutorial/index.html a https://i.iinfo.cz/files/root/k/Ucebnice_jazyka_Python.pdf
\section{QGIS}
\label{sec:qgis}
\begin{figure}[H]
	 \centering
      \includegraphics[height=4cm]{./pictures/qgis-logo.jpg}
      \caption{Logo QGIS (zdroj:
\href{https://euipo.europa.eu/copla/image/CJ4JX4FZVCC523YA2TMALSKFLFPOWZHPVHYMP5QREVP2BOXHB3PCM7RCOZR6TEIMWNCQDAB6N25VA}{qgis.org})}
      \label{fig:qgis}
  \end{figure}
  
  Jedná se o volně dostupný geografický informační systém (\zk{GIS}),
  který slouží pro práci s geodaty\footnote{Data s prostorovou a
    atributovou složkou, která se vztahují ke konkrétnímu místu na
  %% ML: druhou vetu prepiste, aby znela vice cesky
  %% LK: veta zjednodusena
    %% ML: OK
   Zemi.}. Licenci k programu má ve správě GNU General Public
License. QGIS je oficiálním projektem Open Source Geospatial
Foundation(\zk{OSGeo}) a je spustitelný na nejužívanějších operačních
systémech jako Windows, Linux, Mac\-OS. Samotný systém je napsaný v
jazyce C++ a jeho výhodou je velké množství nejrůznějších rozšíření
(zásuvných modulů), které je možné snadno doinstalovat. Zásuvné moduly
mohou být napsané nejen v C++, ale také v programovacím jazyce
Python. QGIS podporuje mnoho formátů -- rastrových,
vektorových i databázových. Aktuálně nejnovější verze je 2.18.15
nazvaná \textit{Las Palmas} a vydaná dne 8.12.2017.

Vývoj systému začal roku 2002 Garym Shermanem, ještě pod názvem
\textit {Quantum GIS} (toto označení zůstalo až do verze 2.0).\cite{qgis_official, qgis_wiki_en, qgis_wiki_cz}
%% ML: tuto informaci mate na zacatku odstavce, staci jednou
%% LK: odstraneno
%% ML: OK
 
%zdroj: https://www.qgis.org/en/site/, https://cs.wikipedia.org/wiki/QGIS, https://en.wikipedia.org/wiki/QGIS
%obrazek: https://euipo.europa.eu/copla/image/CJ4JX4FZVCC523YA2TMALSKFLFPOWZHPVHYMP5QREVP2BOXHB3PCM7RCOZR6TEIMWNCQDAB6N25VA
\section{QGIS VFK Plugin}
\label{sec:qgis_plugin}
\begin{figure}[H]
	 \centering
      \includegraphics[height=5cm]{./pictures/Qgisvfkplugin.png}
      \caption{Ukázka prostředí pluginu (zdroj:
        %% ML: freegis.fsv.cvut.cz
        %% LK: opraveno
\href{http://freegis.fsv.cvut.cz/wiki/images/4/4b/Qgisvfkplugin-screenshot-05.png}{freegis.fsv.cvut.cz})}
      \label{fig:qgis_vfk_plugin}
  \end{figure}

Jde o zásuvný modul (anglicky plugin) pro geografický informační systém
%% ML: predlozka ``z'' neni v tomto kontextu moc vhodna
%% ML: predlozky mate v textu naduzivany a casto ve spatne kontextu (pod/z)
%% LK: na predlozky se zamerim 
QGIS, který umožňuje práci s daty českého katastru
%% ML: na tomto miste bych nepouzival ``uplne'', pro plugin je
%% podstatne, aby se v datech vyskytovali bloky PAR a BUD (skupina
%% NEMO), jinak nemusi byt '``uplna''
%% LK: odstraneno uplnymi, mám zmínit, že data musí obsahovat datovou skupinu NEMO?
%% ML: ano, muzete zminit
%% LK: zmineno
nemovitostí. Zásuvný modul pracuje s daty v takzvaném novém
výměnném formátu katastru, který je označovaný \zk{VFK} nebo
\zk{NVF}. Data ze souboru, který musí obsahovat datovou skupinu NEMO, jsou čtena pomocí knihovny \zk{GDAL}. Plugin
%% ML: Pri cem jinem? ;-) Zkuste preformulovat
%% LK: preformulovano :)
%% ML: OK
umožňuje vyhledávání a zobrazování informací z načtených dat katastru
nemovitostí. Ovládání je uživatelsky přívětivé a známé vzhledem k
podobnému rozhraní jako je u webových aplikací.

%% ML: nyni?
%% LK: zmenen zacatek vety
%% ML: OK
V aktuální verzi je možné pro nahraná data vyhledávat: parcely, budovy, jednotky
a oprávněné osoby. Prohlížeč dat umožňuje zobrazit list vlastnictví a
další výpisy informací: o parcele, o budově, o jednotce a o oprávněné
osobě. Dále prohlížeč umožňuje zobrazení aktuálního stavu nemovitosti
na stránkách Nahlížení do katastru nemovitostí a export výpisů do
formátů HTML a zdrojového kódu LaTeXu (možnost vytvoření PDF a
PS\footnote{Programovací jazyk PostScript vyvinutý ke grafickému
  popisu tisknutelných dokumentů, výhodou je nezávislost na zařízení,
  ze kterého se tiskne, podobně jako formát PDF. \cite{PostScript}}).

%% ML: Obe vety zacinaji temer stejne: Zdrojove kody/texty, zkuste prepsat
%% ML: v poradku
Zdrojové kódy zásuvného modulu jsou ke stažení na adrese
\href{https://github.com/ctu-geoforall-lab/qgis-vfk-plugin}{Git
  repozitáře} a jsou šířeny pod licencí
\href{https://raw.githubusercontent.com/ctu-osgeorel/qgis-vfk-plugin/master/LICENSE}{GNU
  GPL}.

První verze zásuvného modulu (verze1.x) byla napsána v jazyce C++ a vyvinuta studenty
oboru Geoinformatika Annou Kratochvílovou a Václavem Petrášem na FSv
ČVUT v Praze v roce 2012. Druhá verze 2.x byla vyvíjena v letech 2015
a 2016 studentem stejného oboru Štěpánem Bambulou a zásuvný modul byl
přepsán do jazyka Python. \citep{vfk_qgis_plugin}
%http://freegis.fsv.cvut.cz/gwiki/VFK_/_QGIS_plugin
%https://cs.wikipedia.org/wiki/PostScript
 
\section{GDAL}
\label{sec:gdal}
\begin{figure}[H]
	 \centering
      \includegraphics[height=5cm]{./pictures/gdal-logo.png}
      \caption{Logo GDAL (zdroj:
\href{https://upload.wikimedia.org/wikipedia/commons/thumb/d/df/GDALLogoColor.svg/572px-GDALLogoColor.svg.png}{gdal.org})}
      \label{fig:gdal}
  \end{figure}
  
Geospatial Data Abstraction Library (GDAL) je knihovna určená pro čtení
a zápis vektorových a rastrových formátů geodat. Jde o open source
vyvíjený pod licencí X/MIT a jako součást projektu Open Source
Geospatial Foundation (\zk{OSGeo}). Samotná knihovna je reprezentována
jedním abstraktním modelem pro rastrová data a jedním pro vektorová
data. Dále knihovna nabízí užitečné nástroje pro příkazovou řádku,
které slouží ke konverzi a zpracování dat. Od verze GDAL 2.0 je
součástí knihovny GDAL také knihovna OGR, která zajišťuje
funkcionalitu jednoduchých prvků vektorových dat.

Nejdříve byla knihovna vyvíjena Frankem Warmerdamem, od verze 1.3.2
došlo k převedení na GDAL/OGR Project Management Committee, který je
součástí \zk{OSGeo}. Knihovna je díky velké funkcionalitě často
využívána v komerční i nekomerční sféře a proto patří v \zk{GIS} mezi
hlavní volně dostupné softwary. \cite{gdal, gdal_wiki}
%obrazek: https://upload.wikimedia.org/wikipedia/commons/thumb/d/df/GDALLogoColor.svg/572px-GDALLogoColor.svg.png
%zdroje: http://www.gdal.org/, https://cs.wikipedia.org/wiki/GDAL

\section{Qt}

\begin{figure}[H]
	 \centering
      \includegraphics[width=3cm]{./pictures/qt-logo.png}
      \caption{Logo Qt (zdroj:
\href{https://upload.wikimedia.org/wikipedia/commons/thumb/0/0b/Qt_logo_2016.svg/578px-Qt_logo_2016.svg.png}{wikipedia.org})}
      \label{fig:qt}
  \end{figure}

  %% ML: chybi Vam tu kontext: na tomto grafickem frameworku je postaven QGIS!
  %% ML: v poradku
  Qt je multiplatformní aplikační rámec (framework), který je určen
  pro vývoj aplikačního softwaru. Ten může snadno fungovat na různých
  platformách s žádnými nebo jen minimálními změnami v kódu. QGIS (viz
  kap. \ref{sec:qgis}) samotný je na tomto grafickém aplikačním rámci
  postaven. Qt je aktuálně vyvíjen společnostmi \textit{The Qt
    Company} a \textit{Qt Project}.\cite{qt_wiki, qt}

%zdroj: https://en.wikipedia.org/wiki/Qt_(software)
%obrazek: https://upload.wikimedia.org/wikipedia/commons/thumb/0/0b/Qt_logo_2016.svg/578px-Qt_logo_2016.svg.png


\chapter{Knihovna publicvfk}
\label{4-plugin}
Tato kapitola je věnována informacím o nové knihovně
%% ML: v textu Vam chybi obecne provazanost (napr. zde chybi odkaz na
%% kapitolu QGIS VFK Plugin)
\textbf{publicvfk} pro zásuvný modul \textit{QGIS VFK Plugin} a její
%% ML: ve vete mate dvakrat souslovi ``zasuvny modul'', zkuste prepsat
integraci do zásuvného modulu. Je popsána funkčnost knihovny, uvedeno
%% ML: druhou cast vety prepiste, zni sroubovane. - napr. knihovny
%% jeji vstupy a vystupy.
co je pro knihovnu vstupem a co výstupem. Poté je představena ukázka
funkčnosti na testovacích datech a popsány podrobnosti o způsobu
integrace do výše zmíněného zásuvného modulu \textit{QGIS VFK
 %% ML: tvorbu ceho? textu?
  Plugin}. Pro tvorbu je čerpáno ze zdrojů \cite{cookbook,
  ucebnicepython}.

\section{Funkčnost knihovny}
\label{sec:funknost_knihovny}
Knihovna načítá pomocí VFK driveru knihovny GDAL (viz
%% ML: cimz? nejasna veta, zkuste preformulovat
kap. \ref{subsec:gdal_vfk}) textový soubor ve formátu \zk{VFK}, čímž
%% ML: SQL -> SQLite
vznikne \zk{SQL} databáze s načtenými daty. \zk{VFK} soubor již není
%% ML: SQL -> SQLite
dále využíván, knihovna místo toho přistupuje k vytvořené \zk{SQL}
databázi. Následně je zkontrolována verze knihovny
GDAL[\ref{sec:gdal}]. Pokud není vyšší než 2.2, tak musí být proveden
%% ML: prikaz nemate vysvetlen, co dela? - napr. v poznamce pod carou
příkaz \verb|self.dsn_vfk.GetLayerByName('HP').GetFeature(1)|, který
se postará o sestavení geometrie všech datových bloků souvisejících s
blokem hranic parcel (HP)[\ref{sec:sestaveni_geometrie}].

%% ML: nacitani dat z databaze (v GDAL 2.2 je to opet jiz vyreseno)
Pro správné fungování při načítání dat je nezbytné přidat do databáze
tabulku geometrie (\verb|geometry columns|) a tabulku souřadnicového
systému (\verb|spatial_ref_sys|). Dojde tak k vytvoření prostorové
databáze\footnote{Databáze ukládající prostorovou složku dat.}. SQLite
%% ML: bez techto takulek?
%% ML: nez o datovem typu bych hovoril o typu prvku/geometrie
driver knihovny GDAL bez tabulek není schopen rozeznat datový typ vrstev. U novější
%% ML: GDAL 2.2
verze knihovny dochází k vytvoření tabulek automaticky.

%% ML: nemate uplne vysvetleno, proc je nacitani nutne provadet pomoci
%% SQLite driveru. VFK driver totiz take cte data z databaze (ale
%% neuvidi bloky PAR a BUD jelikoz chybi v podkladovem VFK souboru)

Po vytvoření prostorové databáze následují postupně kroky, během
kterých vznikají data v připojené databázi:
\begin{enumerate}[leftmargin=50pt]
\item Vytvoření tabulky s názvem PAR pro parcely
\item Sestavení a zapsání geometrie parcel i s atributy
\item Vytvoření tabulky s názvem BUD pro budovy
\item Sestavení a zapsání geometrie budov včetně atributů
\end{enumerate}

Pro sestavení geometrie parcel je využito datového bloku HP (hranice
parcel), kde je možné pomocí atributů \verb|PAR_ID_1| a
\verb|PAR_ID_2| (viz kap. \ref{subsec:bloky_par_bud}) zjistit unikátní seznam
%% ML: cisel -> identifikatoru
čísel všech parcel. Pro každou parcelu jsou následně nalezeny
%% ML: geometrie sestavena geometrickou cestou? zkuste preformulovat
příslušné hranice. Samotná geometrie parcel je sestavená geometrickou
%% ML: veta nedava smysl, prepiste
cestou, tedy postupným spojováním navazujících hranici parcelu po
parcele. Následující pseudokód(\ref{alg:sestaveni_parcely}) popisuje
proces sestavení a uložení geometrie parcel. Na řádku 10 je volána
metoda \verb|build_bound()|(viz kap. \ref{subsec:sestaveni_geometrie}), která
provádí sestavení geometrie z příslušných hranic. Princip metody je
podrobněji znázorněn diagramem v příloze \ref{fig:logika_geometrie}.

\begin{algorithm}
\caption{Logika sestavení a uložení geometrie parcel}
\label{alg:sestaveni_parcely}
	\begin{algorithmic}[1]
	\STATE{číslaParcel = zjisti SQL příkazem unikátní čísla parcel}
	\STATE{NeuzavřenéParcely = prázdný seznam}
	\STATE{Začátek transakce}
	\FOR{Parcela \textbf{in} číslaParcel}
		\STATE{seznamGeometriíHranice = prázdný seznam}
		\FOR{prvek \textbf{in} filtrVrstvy(vrstva = HraniceParcel, filtr = Parcela)}
			\STATE{geometrie = geometrie prvku}
			\STATE{přidej geometrie do seznamGeometriíHranice}
		\ENDFOR
		\STATE{polygonGeometrie = sestav geometrii ze seznamGeometriíHranice}
		\IF{polygonGeometrie \textbf{is not} prázdný}
			\STATE{převeď polygonGeometrie do roviny(2D)}
		\ELSE
			\STATE{Přidej číslo parcely do NeuzavřenéParcely}
		\ENDIF
		\STATE{Vytvoř nový řádek tabulky}
		\STATE{Nastav geometrii sestavované parcely do nového řádku}
		\STATE{Nastav hodnotu do sloupce \verb|"id_par"| pro nový řádek}
		\STATE{Nastav hodnotu do sloupců \verb|"kmenove_cislo_par"|, \verb|"poddeleni_cisla_par"| pro nový řádek}
		\STATE{Přidej nově vytvořený řádek do tabulky}
	\ENDFOR
	\STATE{Konec transakce}
	\end{algorithmic}
\end{algorithm}

K sestavení geometrie budov je využito datového bloku SBP (spojení
bodů polohopisu) a bloku OB (obrazy budov). Nejprve jsou zjištěna z
bloku OB unikátní identifikační čísla budov včetně příslušných
identifikačních čísel hranic budov, pro které je následně vyhledána
geometrie v tabulce SBP. Sestavení geometrie budov probíhá také
geometrickou cestou. Logika sestavování geometrie je stejná jako v
případě parcel, viz příloha \ref{fig:logika_geometrie}.
%zdroj: http://gdal.org/drv_sqlite.html

\section{Vstupní data}
Vstupními daty je pro knihovnu textový soubor ve formátu \zk{VFK} s
neúplnými daty (viz kap. \ref{subsec:neuplna_data}). Knihovna přebírá
%% ML: adresa -> cesta
adresu vstupního souboru, dochází k načtení dat VFK
driverem (viz kap. \ref{subsec:gdal_vfk}) a zápisu do databáze.

\subsection{Testovací data}
Zkomprimovaná testovací data ve formátu \zk{VFK} byla stažena pro
katastrální území Abertamy na adrese:
\href{http://services.cuzk.cz/vfk/ku/20170901/600016.zip}{http://services.cuzk.cz/vfk/ku/20170901/600016.zip}.
     {\scriptsize
\begin{lstlisting}[caption=Ukázka bloku hranic parcel(HP) -- definice bloků a věty dat(zdroj:vlastní), label=lst:data]
&BHP;ID N30;STAV_DAT N2;DATUM_VZNIKU D;DATUM_ZANIKU D;PRIZNAK_KONTEXTU N1;
RIZENI_ID_VZNIKU N30;RIZENI_ID_ZANIKU N30;TYPPPD_KOD N10;PAR_ID_1 N30;PAR_ID_2 N30
&DHP;3491827403;0;"07.04.2009 08:59:39";"";3;1991606403;;21900;706860403;708070403 
&DHP;3491828403;0;"07.04.2009 08:59:39";"";3;1991606403;;21900;706860403;708070403
&DHP;3491829403;0;"07.04.2009 08:59:39";"";3;1991606403;;21900;706860403;708070403
&DHP;3491830403;0;"07.04.2009 08:59:39";"";3;1991606403;;21900;706860403;708070403
&DHP;3491831403;0;"07.04.2009 08:59:39";"";3;1991606403;;21900;706860403;708070403
\end{lstlisting}}
Na řádcích 1-2~(\ref{lst:data}) je rozdělený uvozovací řádek datového
bloku HP (hranic parcel). Řádky 3-7~(\ref{lst:data}) představují věty
%% ML: poradi datovych vet je v podstate nahodne, je dano exportem z
%% publikacni DB ISKN, ale to pouze na okraj
dat, ve kterých jsou uložena vlastní data ve stanoveném pořadí.

%% ML: Overereni funkcnosti knihovny?
\subsection{Funkčnost knihovny s testovacími daty}

%% ML: prvni veta nedava smysl, databaze neni prostorova (chybi geometry_columns)
%% ML: take nekde chybi prikaz, ktery by ukazoval jak tato DB vznikla (ogrinfo file.vfk)
Přestože databáze obsahuje datové vrstvy, tak SQLite driver není schopen
%% ML: coz je v poradku, jelikoz vstupni data tyto bloky neobsahovala
vrstvy rozeznat, proto mají všechny hodnotu None. Zároveň databáze
neobsahuje bloky parcel a budov.
\begin{figure}[H]
	 \centering
      \includegraphics[height=5cm]{./pictures/funkcnost_knihovny_pred.png}
      \caption{Databáze před použitím knihovny (zdroj:vlastní)}
      \label{fig:funkcnost_pred}
\end{figure}

%% ML: driver se jmenuje SQLite (casta chyba v textu)
Po použití knihovny \textbf{publicvfk} SQL driver datové bloky již
rozezná díky přidaným tabulkám geometrie a souřadnicového systému. V
databázi jsou nyní obsaženy sestavené bloky parcel
\textbf{PAR (Polygon)} a budov \textbf{BUD (Polygon)}.
\begin{figure}[H]
	 \centering
     \includegraphics[height=3cm]{./pictures/funkcnost_knihovny_po.png}
     \caption{Databáze po použití knihovny(zdroj:vlastní)}
     \label{fig:funkcnost_po}
\end{figure}  
  
\section{Výstupní data}
Výstupem knihovny je sestavená geometrie bloků parcel a
budov. Geometrie je společně s atributy zapsána do VFK
driverem (viz kap. \ref{subsec:gdal_vfk}) vytvořené databáze.

\section{Popis tříd knihovny a jejich metod}
\label{sec:popis_trid}
V následující podkapitole jsou představeny jednotlivé třídy knihovny,
jejich členské metody a popsáno, co která třída a metoda obstarává.

\subsection{VFKBuilderError}
Tato třída dědí vlastnosti třídy Exception a je volána v případě, že
%% ML: zadne jine pripadne chybove stavy neexistuji?
nastane chyba. To se může stát není-li připojen \zk{VFK} soubor nebo
databáze.
\subsection{VFKBuilder}
\label{subsec:sestaveni_geometrie}
Mateřská třída, která obsahuje společné metody tříd
\textit{VFKParBuilder} a \textit{VFKBudBuilder} určených pro sestavení
geometrie parcel i budov.
\begin{itemize}[leftmargin=50pt]
\item \verb|__init__()|

%% ML: preteceni textu v radku
V konstruktoru třídy dochází k vytvoření tabulky geometrie
(\verb|geometry columns|) a tabulky souřadnicového systému
(\verb|spatial_ref_sys|), bez kterých by nebylo možné číst geometrii z
databáze. V případě nepřipojeného zdroje dat - \zk{VFK} souboru, je
%% ML: trida neni volana, je vyvolana vyjimka, ktera je obslouzena tridou...
volána třída VFKBuilderError a zobrazena chybová hláška.
\item \verb|build_bound()|

Hlavní metoda, která sestavuje geometrii jednotlivých hranic. V
%% ML: hranice s dirami, nemel jste na mysli polygon?
případě hranice s dírami dojde k vytvoří seznamu s více geometriemi,
%% ML: nejdelsi ? nejvetsi vymera ?
ve kterém je nalezena největší a ze zbylých geometrií jsou vytvořeny
díry. Sestavení probíhá geometrickou cestou. Nejdříve je přidána první
hranice, poté hranice obsahující koncový bod první hranice a tak
%% ML: tady by se hodil odkaz na pseudokod
dokola. Na závěr je otestováno uzavření všech hranic v seznamu
geometrií (první bod hranice je shodný s posledním bodem hranice). Pro
názornost principu metody je v příloze vytvořen
diagram \ref{fig:logika_geometrie}.
\item \verb|add_boundary()|

Metoda pro přidávání jedné hranice do geometrie. Přidávání hranice
probíhá bod po bodu a přidaná hranice je ze seznamu hranic odstraněna,
aby se seznam zmenšil. Všechny hranice nemají stejnou
orientaci (některé na sebe navazují koncovými body), tudíž je potřeba
%% ML: v nekterych pripadech
body hranice přidávat "odzadu".
\item \verb|filter_layer()|

Na základě specifikovaného atributového filtru a názvu datového bloku
vrací výsledné hodnoty uložené v seznamu.
\item \verb|executeSQL()|

Provádí \zk{SQL} dotaz v databázi a vrací výsledek uložený do seznamu.

\end{itemize}
\subsection{VFKBudBuilder}
Potomek třídy \textbf{VFKBuilder}. Třída sestavuje geometrii budov a
ukládá ji do nově vytvořené tabulky BUD v databázi. Ukládání probíhá v
transakci.
\begin{itemize}[leftmargin=50pt]
\item \verb|__init__()|

Konstruktor třídy, kde je vytvořena nová tabulka pro budovy -- BUD a~atribut \verb|id_bud|.
\item \verb|build_all_bud()|

 %% ML: preformulovat, prvni veta je matouci
Metoda se stará o sestavení všech nebo jen části všech budov. To je
možné nastavit parametrem limit. Po sestavení probíhá v transakci
uložení geo\-metrií a atributů do tabulky BUD v databázi.
\end{itemize}
\subsection{VFKParBuilder}
Potomek třídy \textbf{VFKBuillder}. Zde dochází k vlastnímu sestavení
geometrie parcel, vytvoření nové tabulky PAR v databázi a zapsání
dat. Zapisuje se identifikační číslo parcely (\verb|par_id|), kmenové
číslo parcely (\verb|kmenove_cislo_par|), poddělení čísla
parcely (\verb|poddeleni_cisla_par|) a hlavně geometrie dané
parcely. Zápis do databáze je proveden v transakci, čímž je zaručené
korektní zapsání všech parcel.

\begin{itemize}[leftmargin=50pt]
\item \verb|__init__()|

Konstruktor třídy, kde je vytvořena nová tabulka pro parcely -- PAR včetně atributů.
\item \verb|build_all_par()|

Zde probíhá samotné sestavení všech parcel. Po sestavení je parcela
uložena do databáze i s příslušnými atributy. Metodě je možné nastavit
%% ML: Zakladne?
kolik parcel má sestavit. Základně dochází k sestavení všech parcel.

\end{itemize}
\section{Integrace knihovny do zásuvného modulu}
\label{sec:integrace_knihovny}

%% ML: zkuste prvni vetu preformulovat, nezni dobre
Základem integrace je správné umístění do kódu zásuvného modulu,
přesněji do souboru \textit{mainApp.py}. Je potřeba zachovat
%% ML: zpoplatnenych a verejne dostupnych - pripadne upravit i dale v
%% textu (ani zpoplatnena data nejsou uplna...)
funkcionalitu při otevření úplných i neúplných dat. Jsou-li data
úplná, funguje zásuvný modul standardně. Pokud jsou data neúplná --
%% ML: pouziti trid ? zavolani neni korektni termin
neobsahují bloky PAR a BUD, dojde k jejich sestavení a tedy zavolání
třídy z nově integrované knihovny \textbf{publicvfk}.

Nejdříve je knihovna pomocí metody import nahrána. Poté je ve funkci
\textbf{loadVfkFile()} proveden test na přítomnost bloku parcel('PAR')
pomocí metody GetLayerName():
\begin{lstlisting}[language=Python, numbers=none]
t_par = self.__mOgrDataSource.GetLayerByName('PAR')
\end{lstlisting}
Předpokladem je, že bloky parcel a budov jsou v datech obsaženy buďto oba nebo žádný, proto je testována jen přítomnost bloku parcel. Není-li blok obsažen, dojde k uzavření zdroje dat:
\begin{lstlisting}[language=Python, numbers=none]
self.__mOgrDataSource = None
\end{lstlisting}
, aby mohlo proběhnout sestavení bloků parcel a budov. Knihovna si
vytváří vlastní připojení k \zk{VFK} souboru a databázi, proto je
třeba zdroj dat uzavřít a předejít tak zdvojenému připojení k \zk{VFK}
souboru či databázi. Vícenásobné připojení může způsobit chybu. Poté
%% ML: neobsazenych je kostrbate slovo
následuje samotné sestavení neobsažených bloků parcel a budov. Za
%% ML: odkaz na kapitoly popisujici tridy?
sestavení parcel zodpovídá třída \textit{VFKParBuilder} a o sestavení
%% ML: neni to jiz jasne, navic pokud umistite odkaz na popis trid,
%% tak o to bude cela vec jasnejsi
budov třída \textit{VFKBudBuilder}, obě třídy jsou z integrované
knihovny. Nejprve jsou deklarovány objekty dané třídy a následně jsou
volány metody příslušných tříd pro sestavení geometrií:

\begin{lstlisting}[language=Python, numbers=none]
# Build Parcels
parcels = VFKParBuilder(fileName)
parcels.build_all_par()
# Build Buildings
buildings = VFKBudBuilder(fileName)
buildings.build_all_bud()
\end{lstlisting}

Po ukončení sestavování bloků parcel a budov je zdroj dat pomocí
%% ML: nepiste, kde konkteretne DB vznika, to se stejne muze zmenit,
%% pouze muzete zminit, ze plugin pouziva vlastni nastaveni pro DB
%% tak, aby byl schopen nahrat vice VFK soubor ci zpracovat zmenove
%% soubory
proměnné prostředí nastaven na databázi, která vznikne o adresář výš
při otevření \zk{VFK} souboru a nese jméno \zk{VFK} souboru, kde je
místo přípony \verb|.vfk| přípona \verb|_stav.db|: {\small
\begin{lstlisting}[language=Python, numbers=none]
self.__mOgrDataSource = ogr.Open(os.environ['OGR_VFK_DB_NAME'], 0)
self.__mDataSourceName = os.environ['OGR_VFK_DB_NAME']
\end{lstlisting}}
%self.__mDataSourceName = os.environ['OGR_VFK_DB_NAME'] PROČ,k čemu proměnná je? nastavuje zde taky prostredi? 

V této databázi jsou uložena data z načtení \zk{VFK} souboru a také
knihovnou vytvořené tabulky s bloky PAR a BUD. Zásuvný modul z této
databáze čerpá data, která po načtení vizualizuje v mapovém okně.

Při načítání neúplných dat \zk{VFK} může nastat situace, kdy už jsou
tabulky parcel a budov nebo jen jednoho bloku v databázi
zapsané. SQLite driver však bloky nedokáže rozeznat, protože databáze
není prostorová (viz kap. \ref{sec:funknost_knihovny}). Pro tento
případ je v knihovně před vytvářením jednotlivé tabulky testováno,
jestli databáze blok parcel nebo budov opravdu neobsahuje. Tento test
je v kódu umístěn až za vložením tabulek s geometrií a souřadnicovým
systémem, tudíž je nepřítomnost zapsaných dat vyloučena. Zjistí-li se
po přidání tabulek, že jsou oba datové bloky -- parcely a budovy v
databázi již zapsané, knihovna nové sestavení neprovádí.

\section{Testování knihovny}

Funkčnost knihovny je možné otestovat z příkazové řádky. K testování
%% ML: Ta veta je zavadejici, k testovani neni pouzit sys modul, pres
%% sys.argv jste pouze schopen zjisit parametry pri spusteni skriptu
je využit modul sys, který je obsažen v základní distribuci Pythonu a
díky kterému je možné realizovat množství úloh spojených s
interpretrem. Příkaz pro spuštění se skládá z názvu knihovny a
\zk{VFK} souboru včetně přípony, oddělených mezerou. Například:
\textit{python publicvfk.py 600016.vfk} (viz
kap. \ref{fig:testovani_ukazka}). Jméno knihovny a další argumenty
(v~našem případě pouze název \zk{VFK} souboru) předané z příkazové
řádky jsou uloženy v proměnné \textit{sys.argv}, která se chová jako
seznam.

\begin{figure}[H]
	 \centering
      \includegraphics[height=7cm]{./pictures/testovani_ukazka.png}
      \caption{Ukázka testovacího spuštění knihovny}
      \label{fig:testovani_ukazka}
  \end{figure}

Zadání názvu knihovny i názvu \zk{VFK} souboru současně kontroluje
podmínka (\ref{lst:chyba}), pokud je spuštění knihovny nekorektní je
interpret ukončen a zobrazena chybová hláška
(obr. \ref{fig:testovani_hlaska}).
\begin{lstlisting}[caption=Podmínka pro spouštěcí příkaz, language=Python, label=lst:chyba, numbers=none]
    if len(sys.argv) != 2:
        sys.exit("{} soubor.vfk".format(sys.argv[0]))
\end{lstlisting}

\begin{figure}[H]
	 \centering
      \includegraphics[height=7cm]{./pictures/testovani_hlaska.png}
      \caption{Chybová hláška včetně nekompletního příkazu při nesprávném použití knihovny}
      \label{fig:testovani_hlaska}
  \end{figure}

Testování je možné jen při přímém spuštění knihovny, nikoli je-li
knihovna importována jako modul. K tomu je využita speciální proměnná
\verb|__name__|, do které je interpretrem v případě spuštění přímo
uložena hodnota \verb|"__main__"| a podmínka je
splněna (viz.\ref{lst:podminka}). Je-li knihovna importována z jiného
modulu je proměnná \verb|__name__| nastavena na jméno modulu a
podmínka není splněna.
\begin{lstlisting}[caption=Ukázka sestavení bloků provedeném jen při přímém spuštění knihovny, language=Python, numbers=none, label=lst:podminka]
if __name__ == "__main__":
	#Sestaveni bloku primo z knihovny
    parcel = VFKParBuilder(sys.argv[1])
    parcel.build_all_par()
    building = VFKBudBuilder(sys.argv[1])
    building.build_all_bud()
\end{lstlisting}
%ucebnice jazyka Python str 10

\chapter{Závěr}
\label{5-zaver}
%Shrnutí cíle práce
Cílem mojí bakalářské práce bylo vytvoření nové knihovny psané v jazyce Python, která měla rozšířit funkcionalitu již existujícího zásuvného modulu pro aplikaci QGIS usnadňujícího práci s českými katastrálními daty ve formátu \zk{VFK} o načítání neúplných dat výměnného formátu katastru nemovitostí a o sestavení bloků parcel a budov z načtených dat.

%Upřesněný výsledek
Podařilo se vytvořit novou knihovnu, která rozšiřuje zásuvný modul a umožňuje načítání úplných i neúplných dat výměnného formátu katastru nemovitostí. Knihovna dále sestavuje z neúplných dat bloky parcel a budov a ukládá je do databáze, která vznikne při načtení dat \zk{VFK} Driverem knihovny GDAL.

%Komplikace
Při vytváření knihovny se objevilo pár komplikací, které tvorbu lehce zpomalily. Starší verze knihovny GDAL, ze které byl používán VFK Driver, nesestavovala geometrii automaticky přímo při načítání dat, ale až po provedení dotazu na konkrétní geometrii. Dále VFK Driver do vzniklé databáze po načtení dat nepřidával tabulky s geometrií a souřadnicovým systémem, tudíž databáze nebyla prostorová a SQL driver nedokázal rozeznat datové bloky. Tabulky byly do databáze tedy vloženy a SQL Driver poté dokázal bloky přečíst. V nejnovější verzi knihovny jsou oba nedostatky odstraněny. Rozšíření zásuvného modulu bylo vyvíjeno na dvou operačních systémech -- Linux a Windows. Windows se několikrát ukázal jako nevhodný operační systém pro vývoj. Největší komplikace způsobilo použití systémových proměnných, které v prostředí Linuxu fungovalo a v prostředí Windows nikoliv. Situaci vyřešilo až alternativní definování proměnné přes příkaz SetConfigOption().

%Výsledek
Výsledkem bakalářské práce je rozšířený zásuvný modul pro práci s daty katastru nemovitostí ve formátu \zk{VFK} pro volně dostupný geografický informační systém QGIS. Přidané rozšíření umožňuje načít i neúplná data ve formátu \zk{VFK} a sestaví bloky parcel a budov, které jsou v mapovém okně systému QGIS vizualizovány.

%Další vývoj
Funkčnost knihovny byla testována na datech z katastrálního území Abertamy, které obsahuje 1680 parcel a 470 budov. Velikost \zk{VFK} souboru je 6,7 MB. U objemnějších dat trvá sestavování geometrie výrazně déle. Sestavování geometrií by se dalo jistě zrychlit.

!\textbf{Kde budou zdrojové kódy ke stažení!}

%Naplnění očekávání
Práce naplnila moje očekávání a výsledek mě samotného až překvapil. Ze začátku jsem si myslel, že s orientačním během nebude mít práce vůbec nic společného, ale nakonec jsem našel alespoň malou souvislost. Geometrie parcel a budov, kterou lze díky knihovně sestavit pro jednotlivá katastrální území nebo pracoviště, by se nejspíš dala využít jako podklad pro vytváření městských map pro orientační běh.

%Jaké problémy se objevily-operační systémy, jak se naplnila očekávání.
%Shrnout cíl práce a popsat výsledek-co knihovna/zásuvný modul nyní umí.
%Možnost dalšího vylepšení.
(Distribuce zásuvného modulu)?


% Vysázení seznamu zkratek

\begin{seznamzkratek}{ABCDE}

	\novazkratka{PSF}
		  {PSF}
	      {Python Software Foundation}
	      
	 \novazkratka{VFK}
		  {VFK}
	      {Výměnný formát katastru nemovitostí}
	      
	 \novazkratka{NVF}
		  {NVF}
	      {Nový výměnný formát}
	      
	 \novazkratka{ČUZK}
	      {ČUZK}
	      {Český úřad zeměměřický a katastrální}

	 \novazkratka{GIS}
	      {GIS}
	      {Geografický informační systém (Geographic information system)}
	         
	  \novazkratka{GUI}	
	      {GUI}
	      {Grafické uživatelské rozhraní (Graphical user interface)}
	           
	  \novazkratka{S-JTSK}	
	      {S-JTSK}
	      {Systém jednotné trigonometrické sítě katastrální}
	  
	  \novazkratka{ISKN}	
	      {ISKN}
	      {Informační systém katastru nemovitostí}        
	      
	  \novazkratka{GDAL}	
	      {GDAL}
	      {Geospatial Data Abstraction Library}
	      
	  \novazkratka{API}	
	      {API}
	      {Rozhraní pro programování aplikací (Application program interface)}    
	      
	   \novazkratka{OSGeo}	
	      {OSGeo}
	      {Open Source Geospatial Foundation}
	      
	   \novazkratka{PBPP}	
	      {PBPP}
	      {Podrobné polohové bodové pole}

	    \novazkratka{BPEJ}	
	      {BPEJ}
	      {Bonitovaná půdně ekologická jednotka}
	      
	    \novazkratka{RAD}	
	      {RAD}
	      {Rapid Application Development (volně přeloženo rychlý vývoj aplikací)}
	        
\end{seznamzkratek}

% Literatura
\nocite{*}
\def\refname{Literatura}
\bibliographystyle{mystyle} %mystyle(alpha, plain, unsrt, abbrv, czechiso)
\bibliography{literatura}


% Začátek příloh
\def\figurename{Figure}%
\prilohy

% Vysázení seznamu příloh
%\seznampriloh

% Vložení souboru s přílohami
%%%%%%%%%%%%%%%%%%%%%%%%%%%%%%%%%%%%%%%%%%%%%
%%                 PŘÍLOHY                 %%
%%%%%%%%%%%%%%%%%%%%%%%%%%%%%%%%%%%%%%%%%%%%%
\chapter{Přílohy}
\label{prilohy}
\section{Diagram sestavení geometrie hranic}
\begin{figure}[H]
	 \centering
      \includegraphics[width=15cm]{./pictures/Diagram_sestaveni_geometrie_hranic.png}
      \caption{zdroj:vlastní}
      \label{fig:logika_geometrie}
  \end{figure}
  
 \section{Ukázka načtení dat pomocí zásuvného modulu v prostředí QGIS}
 \label{sec:nacteni_dat_ukazka}
 \begin{enumerate}
 \item{Spuštění zásuvného modulu kliknutím na ikonu \texttt{Otevřít prohlížeč VFK}}
  \begin{figure}[H]
	 \centering
      \includegraphics[width=15cm]{./pictures/nacteni_1kr.png}
      \caption{Ikona zásuvného modulu (označena zeleně)}
      \label{fig:1kr_nacteni}
  \end{figure}
  
  \item{Pro výběr VFK souboru stiskněte tlačítko \texttt{Procházet} (označeno zeleně)}
  \begin{figure}[H]
	 \centering
      \includegraphics[width=15cm]{./pictures/nacteni_2kr.png}
      \caption{Výběr VFK souboru}
      \label{fig:2kr_nacteni}
  \end{figure}
  
  \item{Zvolení VFK souboru a výběr kliknutím na tlačítko \texttt{Otevřít}, výběr možný i dvojklikem}
  \begin{figure}[H]
	 \centering
      \includegraphics[width=15cm]{./pictures/nacteni_3kr.png}
      \caption{Načti soubor VFK}
      \label{fig:3kr_nacteni}
  \end{figure}
  
  \item{Kliknutím na tlačítko \texttt{Načíst} se spustí načítání dat}
  \begin{figure}[H]
	 \centering
      \includegraphics[width=15cm]{./pictures/nacteni_4kr.png}
      \caption{Spuštění načítání}
      \label{fig:4kr_nacteni}
  \end{figure}
  
  \item{Probíhá načítání}
  \begin{figure}[H]
	 \centering
      \includegraphics[width=15cm]{./pictures/nacteni_5kr.png}
      \caption{Načítání v procesu, může chvíli trvat}
      \label{fig:5kr_nacteni}
  \end{figure}
  
   \item{Data se po načtení sama zobrazí}
  \begin{figure}[H]
	 \centering
      \includegraphics[width=15cm]{./pictures/nacteni_6kr.png}
      \caption{Načtená data pro obec Vršovka}
      \label{fig:6kr_nacteni}
  \end{figure}
 \end{enumerate}
 
 \section{Ukázka stažení veřejně poskytovaných dat VFK}
 \label{sec:stazeni_dat_ukazka}
  \begin{enumerate}
  \item{Katastrální mapa poskytovaná v různých formátech}
  \begin{figure}[H]
	 \centering
      \includegraphics[width=15cm]{./pictures/stazeni_dat_1kr.png}
      \caption{Výběr dat ve formátu VFK (označeno zeleně)}
      \label{fig:1kr_stazeni}
  \end{figure}
  
  \item{Volba parametru vymezujícího oblast dat}
  \begin{figure}[H]
	 \centering
      \includegraphics[width=15cm]{./pictures/stazeni_dat_2kr.png}
      \caption{Výběr dle katastrálních území(ku) (označeno zeleně)}
      \label{fig:2kr_stazeni}
  \end{figure}
  
  \item{Volba dne vytvoření dat}
  \begin{figure}[H]
	 \centering
      \includegraphics[width=15cm]{./pictures/stazeni_dat_3kr.png}
      \caption{Výběr 1.10.2017 (označeno zeleně)}
      \label{fig:3kr_stazeni}
  \end{figure}
  
  \item{Volba konkrétního katastrálního území, řazeno abecedně}
  \begin{figure}[H]
	 \centering
      \includegraphics[width=15cm]{./pictures/stazeni_dat_4kr.png}
      \caption{Výběr Abertamy - 600016 (označeno zeleně)}
      \label{fig:4kr_stazeni}
  \end{figure}
  
  \item{Stažení a uložení zvolených dat}
  \begin{figure}[H]
	 \centering
      \includegraphics[width=15cm]{./pictures/stazeni_dat_5kr.png}
      \caption{Potvrzení výběru tlačítkem OK (označeno zeleně)}
      \label{fig:5kr_stazeni}
  \end{figure}
  \end{enumerate}

\chapter{Obsah CD}
\label{cd}
 %% ML: zde bude v podstate obsah gitu + pdf

\setlength{\unitlength}{.5mm}
\begin{picture}(250, 220)

  \put(  0, 212){\textbf{.}}

  \put(  1, 200){\line(0, 1){5}}
  \put(  1, 200){\line(1, 0){10} {\textbf{ src}}} 
  \put(150, 200){ zdrojový kód}  

  \put(  1,  190){\line(0, 1){10}}
  \put(  1,  190){\line(1, 0){10} {\textbf{ sample\_data}}}
  \put(150,  190){ testovací data}                     
          
  \put(  1,  180){\line(0, 1){10}}
  \put(  1,  180){\line(1, 0){10} {\textbf{ text}}}
  \put(150,  180){ text práce ve formátu PDF}
      
  \put(  1,  170){\line(0, 1){10}}
  \put(  1,  170){\line(1, 0){10} {\textbf{ zadani}}}
  \put(150,  170){ zadání bakalářské práce}
\end{picture}

% Konec dokumentu
\end{document}
