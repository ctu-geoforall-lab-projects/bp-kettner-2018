\chapter{Úvod}
\label{1-uvod}
%Motivace a výběr tématu
Hlavní motivací při výběru tématu bakalářské práce pro mě byla další využitelnost jejího výsledku. Nejdříve jsem si práci pro mě s naprosto novým programovacím jazykem Python ozkoušel na projektu zásuvného modulu pro práci s registrem územní identifikace, adres a nemovitostí( \textit{qgis-ruian-plugin}). Práce se mi zalíbila natolik, že jsem se rozhodl ubírat svojí bakalářskou práci směrem zásuvného modulu pro QGIS. Lákalo mě vyzkoušet si naprogramovat něco užitečného a využitelného. Zároveň jsem chtěl, abych si během tvorby bakalářské práce udělal jasnější obraz o tom, jakým směrem bych se v dalším studiu chtěl vydat.

%Očekávání
Začátek není nikdy jednoduchý, s tím do tvorby práce také jdu. Chci se hlavně naučit něco nového, což si zajisté splním. Očekávám, že programování nové knihovny bude v hodně věcech podobné programování výpočetních skriptů, které mě provázelo celým bakalářským studiem. Řešení logických problémů a hledání výpočetních chyb už mám mnohokrát za sebou a věřím, že se mi to bude při psaní knihovny hodit. I když je jasné, že komplikace při vzniku práce se mohou objevit kdekoliv. Věřím, že vývoj dalšího rozšíření půjde o něco lépe vzhledem k předchozí zkušenosti se zásuvným modulem.

%Předtavení zadaní
V roce 2016 vnikl v rámci diplomové práce zásuvný modul \textit{QGIS VFK Plugin} pro QGIS, který umožňuje práci s katastrálními daty. Tento zásuvný modul vychází z diplomové práce \textit{Quantum GIS plugin for Czech cadastral data}, jejíž zásuvný modul rozšiřuje o další funkcionalitu a přepisuje do jazyka Python pro snadnější distribuci zásuvného modulu.

%Cíl práce
Mým úkolem bude tento zásuvný modul v jazyce Python ještě rozšířit o novou knihovnu, která bude umožňovat nahrání, prohlížení a vyhledávání i v neúplných a volně poskytovaných nekomerčních datech. Rozšíření bude zaměřené na sestavení bloků PAR a BUD (parcel a budov), které v těchto datech nejsou sestaveny. Jedná se o dva nejdůležitější bloky pro vizualizaci v grafickém prostředí. Přestože data dané bloky neobsahují, je možné bloky parcel a budov z dat sestavit. Výsledkem práce by měla být implementace knihovny do již existujícího zásuvného modulu a jeho úprava, aby byla zachována co největší funkčnost zásuvného modulu. Zároveň se budu snažit dostat z neúplných dat co nejvíc informací, abych využil většinu dostupných dat a výsledek se co nejvíce podobal datům úplným.

%Struktura práce
Práce bude logicky uspořádána do dvou celků. Jedním bude teoretická část zabývající se představením informačního systému katastru (\zk{ISKN}) a jeho historie. Dále bude rozebrán výměnný formát katastru (\zk{VFK}), ve kterém jsou data poskytována. Přidám porovnání úplného a neúplného formátu, doplněné o přehled datových bloků, které jsou nezbytné pro sestavení geometrie prvků digitální katastrální mapy. Uvedu také další softwary pro čtení a práci s daty \zk{VFK}. V teorii nebude chybět ani použitá technologie pro vznik bakalářské práce jako například knihovna GDAL, programovací jazyk Python a VFK driver.

Druhá část už bude zaměřené na praktickou část práce. Podrobně představím funkcionalitu jednotlivých tříd z nově vzniklé knihovny. Stěžejní částí bude způsob implementace nově vzniklé knihovny do již existujícího pluginu, která bude doplněná o ukázky funkčnosti.

%%%POZNAMKY
%-Data výměnného formátu katastru(\zk{VFK}) jsou poskytována ve dvou podobách.

%Rešerše% \textbf{Rešerše:} 
%Odkaz v textu% \footnote{\url{http://theses.cz/id/o3vhp8/Diplomov_prce_Lokov.pdf}}
%Kurzíva% \textit{}
%Zkratka% \zk{ZHN}
%\footnote{}