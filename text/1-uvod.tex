\chapter{Úvod}
\label{1-uvod}

Hrozba jaderného výbuchu či jaderné havárie se v
posledních sto letech stala více než reálnou. Proto ve světě vznikla
potřeba být na takovéto situace co nejlépe připraven. V první řadě
existuje samozřejmě snaha jim předcházet, avšak pokud již některý 
ze~zmíněných stavů nastane, je důležité na něj reagovat rychle a
efektivně.  Důsledkem jsou snahy o zjednodušení získávání informací,
automatizaci jejich zpracování a standardizaci formátu, v němž jsou
předávány navazujícím složkám soustavy. Jedním ze způsobů, jak do
%%% ML: slovo obrovsky by bylo dobre nahradit vhodnejsim ekvivalentem
tohoto velkého a provázaného systému přispět, je i softwarový
nástroj, jehož vytvoření v rámci této bakalářské práce zadala Armáda
České republiky, přesněji 314. centrum výstrahy proti zbraním
hromadného ničení v Hostvici.

314. centrum výstrahy \zk{ZHN} je podřízeno 31. pluku radiační,
chemické a bio\-logické ochrany v Liberci. V Armádě ČR plní funkci
související se sledováním a~vyhodnocováním informací v oblasti
radiační, chemické a biologické ochrany.

V období míru je úkolem 314. centra mimo jiné spravovat armádní
radiační monitorovací síť Armády České republiky, provádět letecký
radiační průzkum, shromažďovat informace o zbraních hromadného ničení,
jaderných energetických zařízeních a navrhovat ochranná opatření proti
\zk{ZHN} či ochranu proti následkům radiačních havárií. "Při vyhlášení
stavu ohrožení státu/válečného stavu přebírá centrum od Ministerstva
vnitra úkoly ústředního koordinačního orgánu v oblasti moni\-torování a
výstrahy". \cite{ZHN}

V současnosti AČR zpracovává hodnoty naměřené v rámci radiačního
průzkumu ručně, za výpočty a zakreslení výsledků do mapy je zodpovědná
analytická skupina. Tento proces je poměrně náročný na znalosti a
zkušenosti operátora, ani čas strávený vyhodnocením není
zanedbatelný. Výhodný by proto byl softwarový nástroj, který by část
procesu zautomatizoval. Konkrétně se jedná o vytvoření
předdefinovaných izolinií ze vstupního interpolovaného gridu, jejich
převod na zjednodušené polygony a vygenerování textového reportu ve
%%% ML: souradnicovy nebo hlasny?
%%% TK: spravne ma byt hlasny
formátu dle specifikace \zk{NATO}/\zk{AČR} v hlásném systému
\zk{MGRS}. Na operátorovi pak zůstane následné vložení zprávy do
softwaru určeného k varování a uvědomování ostatních jednotek.

S ohledem na skutečnost, že \zk{AČR} zpracovává data v open source
geografickém informačním systému QGIS, bylo rozhodnuto, že nástroj
bude vyvíjen jako nový zásuvný modul pro toto prostředí. Modul bude
psán v programovacím jazyku Python, pro grafické rozhraní bude použit
framework Qt, bude využívat QGIS \zk{API}. Pro své specializované
funkce je uvažováno i využití GRASS GIS \zk{API} a knihovna \zk{GDAL}.

V teoretické části práce bude čtenář seznámen se způsoby monitorování
radiační situace v České republice, dotkne se tématu standardizovaných
%%% ML: hlásný ?
zpráv předávaných v rámci armády a představí hlásný systém \zk{MGRS}.

\textbf{Rešerše:} Z českých prací se zadanému tématu z hlediska teorie
nejvíce blíží diplomová práce Bc. Romany Loškové \textit{Hodnocení
přístrojů používaných v AČR v~případě mimořádné radiační
události}\footnote{\url{http://theses.cz/id/o3vhp8/Diplomov_prce_Lokov.pdf}}. Zabývá
se monitorováním radiační situace a~přístroji používanými v rámci
\zk{AČR}. Naopak z hlediska praktické části je nejpodobnější
bakalářská práce Ondřeje Peška \textit{Posun letecky měřených bodů po
  trajektorii v prostředí
  %%% ML: nahradil jsem neoficialnim odkazem, ktery je uchopitelnejsi 
QGIS}\footnote{\url{https://github.com/ctu-osgeorel-proj/bp-pesek-2016/raw/master/text/ondrej-pesek-bp-2016.pdf}},
která byla vedena v roce 2016 na katedře geomatiky FSv ČVUT. Jejím
cílem je také vytvoření zásuvného modulu pro QGIS.