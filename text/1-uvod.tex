\chapter{Úvod}
\label{1-uvod}
%Úvod úvodu
Téma bakalářské práce jsem si chtěl vybrat pořádně. Zvolit jsem si chtěl takové téma, které mě zaujme, něco nového se naučím a výsledek bude ideálně dál využitelný. Proto bylo jasné, že o orientačním běhu, který je mojí srdeční záležitostí, psát nebudu, poněvadž tím bych moc nového neobjevil. Výběr tématu jsem začal řešit už v souvislosti s možným výjezdem do zahraničí, kam jsem chtěl během zimního semestru vyrazit. To se po návštěvě studijního oddělení ukázalo jako značně komplikované a proto jsem se rozhodl zůstat s prací na domácí fakultě.

%Proces výběru
Během bakalářského studia mě bavilo programování výpočetních skriptů a taky se mi zalíbil geografický informační systém, na který jsme měli ve třetím ročníku samostatný předmět. Proto jsem s bakalářskou prací zamířil na katedru geomatiky. První téma, které jsem dostal na vyzkoušení, byla tvorba zásuvného modulu pro práci s registrem územní identifikace, adres a nemovitostí. Projekt se jmenoval \textit{qgis-ruain-plugin} a poprvé jsem si zde vyzkoušel práci s programovacím jazykem Python. Za úkol jsem měl osahat si prostředí zásuvného modulu a trošku změnit zdrojový kód. Během testování jsem si samostudiem osvojil základy programovacího jazyka.

%Budoucnost
Práce mě zaujala natolik, že jsem se rozhodl ubírat tvojí bakalářskou práci směrem zásuvného modulu pro geografický informační systém QGIS. Lákalo mě vyzkoušet si naprogramovat něco užitečného a funkčního. Zároveň bych si chtěl během tvorby bakalářské práce udělat jasnější obraz o tom, jakým směrem by se mohla moje studia ubírat. Stále nemám jasnou volbu mezi geomatikou a inženýrskou geodézií. Něco mě táhne k geomatice, tak doufám, že mi to bakalářská práce potvrdí.

%Motivace
Začátek nebude jednoduchý, s tím do tvorby práce jdu. Očekávám, že programování nové knihovny, která je základním kamenem rozšíření funkcionality zásuvného modulu, bude v hodně věcech podobné psaní výpočetních skriptů.  Věřím, že zúročím zkušenosti jak z výpočetních skriptů, tak ze zkušební práce na zásuvném modulu a práce mi půjde dobře od ruky.

%Představení zadaní
Jako téma práce jsem si zvolil \textit{Rozšíření zásuvného modulu QGIS pro práci s katastrálními daty o podporu veřejně dostupných dat ve formátu VFK}.  Moje práce bude dál rozšiřovat funkčnost zásuvného modulu z diplomové práce Bc. Štěpána Bambuly. \textit{Rozšíření nástroje pro práci s katastrálními daty v programu QGIS}. Práce pana Bambuly navazovala na již existující nástroj a rozšířila ho o zpracování a vizualizaci datových vět změnových souborů \zk{VFK}. Dále nástroj přepsal do jazyka Python, aby usnadnil distribuci zásuvného modulu v prostředí geografického informačního systému QGIS.

%Cíl práce
Cílem mojí práce bude tento zásuvný modul v jazyce Python ještě rozšířit o novou knihovnu, která umožní nahrání, prohlížení a vyhledávání i v neúplných a veřejně dostupných datech ve formátu \zk{VFK}. Rozšíření bude zaměřené na sestavení bloků PAR a BUD (parcel a budov), které v těchto datech nejsou obsaženy, ale mohou být z přítomných dat sestaveny. Jedná se o dva nejdůležitější geometrické bloky, které jsou nezbytné pro vizualizaci dat a tvoří datový blok nemovitostí. Po vytvoření knihovny by měla následovat integrace do již existujícího zásuvného modulu. Budu se maximálně snažit využít veškeré v datech obsažené informace, aby se výsledek co nejvíce podobal datům úplným. Tím dojde i k zachování většiny funkcí rozšiřovaného zásuvného modulu.

%Struktura práce
Samotná práce bude logicky uspořádána do dvou celků. Prvním bude teoretická část zabývající se představením informačního systému katastru nemovitostí(\zk{ISKN}) a jeho historie. Dále bude rozebrána základní struktura výměnného formátu katastru (\zk{VFK}), ve kterém jsou data poskytována. Přidám porovnání úplného a neúplného formátu doplněné o přehled datových bloků, pro které je nutné sestavit geometrii, aby šlo bloky PAR a BUD knihovnou také sestavit. Představím použitou technologii, kam patří například programovací jazyk Python a knihovnu GDAL.

Druhá část už bude zaměřená čistě na praktickou stránku práce. Podrobně představím funkcionalitu jednotlivých tříd a členských metod z nově vzniklé knihovny a také funkčnost knihovny samotné. Součástí praktické části bude i integrace vzniklé knihovny do již existujícího zásuvného modulu, která bude také představena a doplněna o ukázky načítání dat.

%%%POZNAMKY
%-Data výměnného formátu katastru(\zk{VFK}) jsou poskytována ve dvou podobách.

%Rešerše% \textbf{Rešerše:} 
%Odkaz v textu% \footnote{\url{http://theses.cz/id/o3vhp8/Diplomov_prce_Lokov.pdf}}
%Kurzíva% \textit{}
%Zkratka% \zk{ZHN}
%\footnote{}