\chapter{Úvod}
\label{1-uvod}
%Úvod úvodu
Tato bakalářská práce má za cíl rozšíření již existujícího
softwarového nástroje (tzv. zásuvného modulu) programu QGIS, který umožňuje
uživatelům práci s českými katastrálními daty poskytovanými ve výměnném
%% ML: slovo uplna bych navrhoval vynechat
%% LK: vynechano
formátu (VFK). V současné době zásuvný modul načítá pouze
zpoplatněná data, ve kterých je možné vyhledávat a prohlížet vybrané
parcely a~budovy. Aktuální verze zásuvného modulu dále nabízí
zpracování vět změnových souborů a nahlížení do katastru nemovitostí
přes internetovou službu poskytovanou Českým úřadem zeměměřickým a
katastrálním.
%% ML: prvni veta (vybrat poradne) je naprosto nevhodna, jde o
%% vedeckou praci, nejste v hospode ;-), nutne prepsat ML: cely prvni
%% odstavec zni kostrbate, text se Vam rozpada po stylisticke i
%% logicke strance, zkuste cely odstravec prepsat
%% LK: odstavec kompletne predelan
%% ML: text jsem mirne revidoval, mate tam jeste jednu novou poznamku, jinak OK
%% LK: jak mám text psát, aby měl stejnou velikost jako po vašem přečtení?
%% nechci přidělávat práci
%% ML: zalezi na editoru, nekde v nastaveni lze nastavit delku textoveho radku, jaky editor konkretne pouzivate?
%% LK: pouzivam Texmaker

%Proces výběru
%% ML: osobni pribeh vynechte, pro volbu tematu nemel az tak zasadni
%% vlic
%% LK:odstavec odstranen
%% ML: stacilo zakomentovat, ale to nevadi

%Budoucnost
%% ML: Prvni vetu zkuste prepsat (ubirat svoji)
%% ML: mozna prilis osobni
%% LK: odstavec odstraněn
%% ML: stacilo zakomentovat, ale to nevadi

%Motivace
%% ML: tady osobni nadech jeste eskaluje, zkuste prepsat ML: uvod
%% pisete v budoucim case, to neni uplne vhodne, text vznika behem
%% prace a nikoliv v minulosti
%% LK: odstavec odstranen
%% ML: stacilo zakomentovat, ale to nevadi

%% ML: nejsem si jist, zda bylo nutne vse odstranovat, v nejake podobe
%% cast textu mohl zustat. Mozna jsem Vas prilis nabudil tim, ze uvodu
%% neuskodi osobni raz, ale ne v takoveto mire. Odstranenim celeho
%% textu se zase vytratit temer uplne, s tim uz nic neudelame, neni cas
%% LK: bude to tak lepsi, tezko bych hledal optimalni miru osobniho razu

%Návaznost práce
Rozšiřovaný zásuvný modul \textit{VFK Plugin} vznikl v rámci diplomové
práce Bc. Štěpána Bambuly \textit{Rozšíření nástroje pro práci s
  katastrálními daty v programu
  QGIS}\footnote{https://github.com/ctu-geoforall-lab-projects/dp-bambula-2016}. Práce
pana Bambuly navazovala na již existující nástroj a rozšířila ho o
zpracování a vizualizaci datových vět změnových souborů
\zk{VFK}. Součástí práce bylo i přepsání kódu nástroje do jazyka
Python, aby usnadnil distribuci zásuvného modulu v prostředí
geografického informačního systému QGIS.

%Cíl práce
%% ML: zvazte prepsani celeho textu do pritomneho casu (Cilem teto
%% prace je...)
%% LK: urcite vyrazne lepsi
%%
%% ML: spojka 'a' evokuje pocit, ze jde o dve formy, data mohou byt
%% 'neuplna' a take 'verejna': data verejne neobsahuji vsechny datove
%% bloky v porovnani s neverejnymi daty, ktera jsou poskytovana za
%% uplatu
%% LK: Odstavec jsem přepsal
%%
%% ML: nemluvil bych o 'geometrickych blocich' ale datovych blocich s
%% geometrickou sloznou popisu
%% LK:ano, souhlasim
%% ML: lepsi, az na ty dve posledni vety
Cílem bakalářské práce je vytvoření nového rozšíření zásuvného modulu
\textit{VFK Plugin} v podobě knihovny, která nabídne uživatelům
možnost načítat i neúplná veřejně dostupná data výměnného formátu
katastru poskytovaná bezplatně Českým úřadem zeměměřickým a
katastrálním.  To znamená sestavit dva, pro vizualizaci
nejdůležitější, datové bloky s geometrickou složkou popisu PAR
%% ML: nejsou soucasti verejne poskytovanych dat (nejen bloky PAR a
%% BUD ale i mnoho dalsich ze skupiny NEMO, ale to pouze na okraj, zde
%% to nemusite uvadet
%% LK: dobre
(parcely) a BUD (budovy). Tyto datové bloky nejsou součástí veřejně poskytovaných
dat, nicméně přesto je možná je z obsažených informací
%% ML: veta Vzniklou ... mi nedava prilis smysl, zkuste jeste
%% preformulovat ;-)
%% LK: vytvorenou?
vytvořit. Vytvořenou knihovnu začlením do rozšiřovaného zásuvného modulu
%% ML: posledni veta mi tez nedava prilis smysl, zkuste preformulovat
%% LK: zkusil jsem, pokud to není lepši, odstraním
se snahou zachovat pokud možno jeho původní funkčnost. Za tímto účelem
maximálně vyžiji informací veřejně poskytovaných dat.

%Struktura práce
%% ML: po presani do pritomneho casu, muzete pridat odkazy na
%% jednotlive kapitoly
%% LK:v pritomnem case
Samotná práce je logicky uspořádána do dvou celků. Prvním je
teoretická část zabývající se představením informačního systému
katastru nemovitostí(\zk{ISKN}) (viz kap. \ref{2-teorie}) a jeho
historie. Dále je rozebrána základní struktura výměnného formátu
katastru (kap. \ref{sec:struktura_vfk}), ve kterém jsou data
poskytována a přidáno porovnání úplného a neúplného formátu
(kap. \ref{subsec:neuplna_data}).  Porovnání je doplněné o přehled
datových bloků, pro které je nutné sestavit geometrii, aby šlo bloky
PAR a BUD knihovnou také sestavit
(kap. \ref{sec:sestaveni_geometrie}). V závěru teoretické části
představuji použitou technologii, kam patří například programovací
jazyk Python (kap. \ref{sec:python}) a knihovna GDAL
(kap. \ref{sec:gdal}).

Druhá část je již zaměřená čistě na praktickou stránku práce. Zde je
podrobně představena funkcionalita jednotlivých tříd a členských metod
(kap. \ref{sec:popis_trid}) z nově vzniklé knihovny a také funkčnost
knihovny samotné (kap. \ref{sec:funknost_knihovny}). Nedílnou součástí
praktické části je jak samotná integrace vzniklé knihovny do zásuvného
modulu, tak její podrobné představení doplněné o ukázky načítání dat
(kap. \ref{sec:integrace_knihovny}).
