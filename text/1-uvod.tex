\chapter{Úvod}
\label{1-uvod}


%Úvodní slovo a předtavení zadaní
Před dvěma roky vnikl v rámci diplomové práce zásuvný modul\textit{QGIS VFK Plugin} pro QGIS, který umožňuje práci s katastrálními daty. Mým úkolem bude tento zásuvný modul rozšířit o další funkcionalitu, kterou je zpracování volně poskytovaných nekomerčních dat. Rozšíření bude zaměřené na sestavení bloků PAR a BUD (parcel a budov), které v těchto datech nejsou obsaženy a jsou stěžejní pro vizualizaci v grafickém prostředí. Bloky sice nejsou v datech přímo obsaženy, ale z dostupných geometrických a popisných dat je možné bloky sestavit.

%Struktura práce
Cílem bakalářské práce tedy bude vytvořit rozšíření stávajícího zásuvného modulu QGIS a implementovat ho. Rozšíření bude psáno v jazyce Python s využitím knihovny GDAL. 

V teoretické části práce představím informační systém katastru nemovitostí(ISKN) včetně málo historie, přiblžím vývoj výměnného formátu katastru a uvedu
použité technologie - Python, knihovna GDAL, QGIS, GDAL-VFK Driver.

V praktické části představím rozšíření zásuvného modulu z praktického i teoretického hlediska. Uvedu ukázky funkčnosti modulu a přidám podrobný popis naprogramovaných tříd a jejich metod.

Výstupem práce bude implementované rozšíření stávajícího zásuvného modulu.

%%%POZNAMKY
%-Data výměnného formátu katastru(\zk{VFK}) jsou poskytována ve dvou podobách.

%Rešerše% \textbf{Rešerše:} 
%Odkaz v textu% \footnote{\url{http://theses.cz/id/o3vhp8/Diplomov_prce_Lokov.pdf}}
%Kurzíva% \textit{Posun letecky měřených bodů po trajektorii v prostředí QGIS}
%Zkratka% \zk{ZHN}