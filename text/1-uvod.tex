\chapter{Úvod}
\label{1-uvod}
%Úvod úvodu
Tato bakalářská práce má za cíl rozšíření již existujícího zásuvného modulu programu QGIS, který umožňuje uživatelům práci s českými katastrálními daty výměnného formátu katastru. V současné době zásuvný modul načítá pouze úplná a zpoplatněná data, ve kterých je možné vyhledávat a prohlížet vybrané parcely a budovy. Dále aktuální verze zásuvného modulu nabízí zpracování vět změnových souborů a nahlížení do katastru nemovitostí přes internetovou službu Nahlížení do katastru nemovitostí poskytovanou Českým úřadem zeměměřickým a katastrálním .
%% ML: prvni veta (vybrat poradne) je naprosto nevhodna, jde o
%% vedeckou praci, nejste v hospode ;-), nutne prepsat ML: cely prvni
%% odstavec zni kostrbate, text se Vam rozpada po stylisticke i
%% logicke strance, zkuste cely odstravec prepsat
%% LK: odstavec kompletne predelan
%% LK: jak mám text psát, aby měl stejnou velikost jako po vašem přečtení?
%% nechci přidělávat práci

%Proces výběru
%% ML: osobni pribeh vynechte, pro volbu tematu nemel az tak zasadni
%% vlic
%% LK:odstavec odstranen

%Budoucnost
%% ML: Prvni vetu zkuste prepsat (ubirat svoji)
%% ML: mozna prilis osobni
%% LK: odstavec odstraněn

%Motivace
%% ML: tady osobni nadech jeste eskaluje, zkuste prepsat ML: uvod
%% pisete v budoucim case, to neni uplne vhodne, text vznika behem
%% prace a nikoliv v minulosti
%% LK: odstavec odstranen

%Návaznost práce
Rozšiřovaný zásuvný modul \textit{VFK Plugin} vznikl v rámci diplomové práce Bc. Štěpána Bambuly \textit{Rozšíření nástroje pro práci s katastrálními daty v programu QGIS}.  Práce pana Bambuly navazovala na již existující nástroj a rozšířila ho o zpracování a vizualizaci datových vět změnových souborů \zk{VFK}. Dále nástroj přepsal do jazyka Python, aby usnadnil distribuci zásuvného modulu v prostředí geografického informačního systému QGIS.

%Cíl práce
%% ML: zvazte prepsani celeho textu do pritomneho casu (Cilem teto
%% prace je...)
%% LK: urcite vyrazne lepsi
%%
%% ML: spojka 'a' evokuje pocit, ze jde o dve formy, data mohou byt
%% 'neuplna' a take 'verejna': data verejne neobsahuji vsechny datove
%% bloky v porovnani s neverejnymi daty, ktera jsou poskytovana za
%% uplatu
%% LK: Odstavec jsem přepsal
%%
%% ML: nemluvil bych o 'geometrickych blocich' ale datovych blocich s
%% geometrickou sloznou popisu
%% LK:ano, souhlasim
Cílem bakalářské práce je vytvoření nového rozšíření zásuvného modulu {VFK Plugin} v podobě knihovny, která nabídne uživatelům možnost načítat i neúplná, veřejně dostupná data výměnného formátu katastru poskytovaná bezúplatně Českým úřadem zeměměřickým a katastrálním.  To znamená sestavit dva, pro vizualizaci nejdůležitější, datové bloky s geometrickou složkou popisu PAR (parcely) a BUD (budovy), které nejsou součástí neúplných dat. Přesto je možné datové bloky z obsažených informací vytvořit. Vzniklou knihovnu začlením do rozšiřovaného zásuvného modulu se snahou zachovat pokud možno původní funkčnost. Z neúplných dat proto využiji maximum informací, aby zůstala funkční co největší část nástrojů modulu.

%Struktura práce
%% ML: po presani do pritomneho casu, muzete pridat odkazy na
%% jednotlive kapitoly
%% LK:v pritomnem case
Samotná práce je logicky uspořádána do dvou celků. Prvním je
teoretická část zabývající se představením informačního systému
katastru nemovitostí(\zk{ISKN})[\ref{2-teorie}] a jeho historie. Dále je rozebrána
základní struktura výměnného formátu katastru[\ref{sec:struktura_vfk}], ve kterém
jsou data poskytována. Přidáno porovnání úplného a neúplného formátu[\ref{subsec:neuplna_data}]
doplněné o přehled datových bloků, pro které je nutné sestavit
geometrii, aby šlo bloky PAR a BUD knihovnou také sestavit[\ref{sec:sestaveni_geometrie}]. Představím
použitou technologii, kam patří například programovací jazyk Python[\ref{sec:python}] a
knihovna GDAL[\ref{sec:gdal}].

Druhá část už je zaměřená čistě na praktickou stránku
práce. Podrobně představím funkcionalitu jednotlivých tříd a členských
metod[\ref{sec:popis_trid}] z nově vzniklé knihovny a také funkčnost knihovny
samotné[\ref{sec:funknost_knihovny}]. Součástí praktické části je dále integrace vzniklé knihovny do
již existujícího zásuvného modulu,která je také představena a
doplněna o ukázky načítání dat[\ref{sec:integrace_knihovny}].
